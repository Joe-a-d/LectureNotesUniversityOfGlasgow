\section{Introduction}

    \par{The main goal of this course is to introduce students to :}
        \begin{itemize}
            \item \ita{critical} research techniques by actively engaging with the materials;
            \item core scientific principles and methods so that one is confident that one's
findings are valid
            \item experimental design and statistical analysis;
            \item planning and management techniques;
            \item methods on how to structure theses and papers
        \end{itemize}
\subsection{ILOs}
    \subsubsection{Critical Review}
        \par{Research papers are the main method for communicating methods and findings and by
critically reviewing a paper you review both the \ita{quality} and \ita{relevance} of the research.
This, in turn, will help inform the direction of your own research}

    \subsubsection{Literature Search}
        \par{In this module you'll learn how to search for \ita{relevant} papers, with the emphasis
being on performing \ita{comprehensive} literature surveys so that you can be confident that you
have a broad understanding of a research are}

    \subsubsection{Literature Survey}
        \par{After the search you'll want to try and synthesize all the papers you deemed relevant,
this will not only serve as reference material which you can revisit but it will also help others by
providing context to your own work}

    \subsubsection{Experiment Design}
        \par{The goal of this module is to educate you on how to conduct good experiments, so that
you can gather valid data}

    \subsubsection{Project Proposal}
        \par{Given that research requires resources it is important that you argue for why other
people should invest in your project}

    \subsubsection{Analysis of Experiment Designs}
        \par{In order to be able to critically evaluate a published experiment it is important that
you're able to break it down and fully understand it, even (in some instances) to be able to
replicate it}

    \subsubsection{Controls \& Sample Size}
        \par{Key concepts which prevents you from  adding bias to your conclusions and allows for
them to be generalisable}

    \subsubsection{Ethics}

\subsection{Research Activities and Outcome}

    \par{Research consists of a lot of different activities, but independently of the field or scale of your project the following always take place:}


    $$\begin{array}{llll}\text { \textbf{Activity} } & \text {\textbf{Notes}} & \text
{\textbf{Outcome}} \\ \text { Read
and understand a paper } & \text { Read thoroughly and more than once } & \text { Understanding of
the work } \\ \text { Summarise a paper } & \text { Demonstrate that you understand it; } & \text {
Annotated bibliography, notes } \\ & \text { its key points and conclusions } & \\ \text { Explain
how different papers } & \text { Agreements, contradictions, } & \text { Literature review } \\
\text { relate to each other } & \text { extensions, replications } & \\ \text { Identify a gap in
the literature } & \text { Is it useful, relevant, important? } & \text { Research idea } \\ \text {
Identify a research question } & \text { Must be a clear and answerable } & \text { Research
question(s) } \\ \text { arising from that gap } & \text { question } \\ \text { Develop a research
plan to } & \text { Demonstrate feasibility-can you do it } & \text { Research proposal } \\ \text {
address the research question } & \text { and do you have the right resources? } & \\ \text {
Execute the research plan } & \text { Multi-faceted. Collaborate with } & \text { Data, frameworks,
models, } \\ & \text { others, supervise, learn more. } & \text { systems, algorithms, etc. } \\
\text { Write and publish papers } & \text { Journals, conference proceedings, etc. } & \text {
Papers and presentations } \\ \text { Critically review papers } & \text { Peer review, discussing
papers, etc. } & \text { Critique, suggestions, decision }\end{array}$$


