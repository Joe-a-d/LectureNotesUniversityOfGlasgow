
%% CLASS MANUAL FOUND IN http://blog.poormansmath.net/latex-class-for-lecture-notes/ %%
%% CLASS AUTHOR Stefano Maggiolo %%

%%%%%%%%%%%%%%%  TITLE PAGE  %%%%%%%%%%%%%%%%%%%
\documentclass[english,course]{Notes}
\title{Research Methods \& Techniques}
\author{Joao Almeida-Domingues}
\email{2334590D@student.gla.ac.uk}
\speaker{Dr.Euan Freeman}
\date{29}{09}{2020}
\dateend{10}{11}{2020}
\place{University of Glasgow}


%%%%%%%%%%%% BIB CONFIG %%%%%%%%%%%%%%%
\usepackage[backend=biber, style=reading]{biblatex} 

\bibliography{} %add bib file name

%%%%%%%%%%% LAYOUT  %%%%%%%%%%%%%%%

\renewcommand{\abstractname}{\vspace{3\baselineskip}} %hack to remove abstract


%%%%%%%%%%%%%%%%%%%%%%%%%%%%%%%%

%%%%%%%%%%%%%% KEEP HERE (conflict when in class) %%%%%%%%%%%%%%%%%%%%

 %%%%%MATRICES
    
    \let\mat=\spalignmat
    \let\amat=\spalignaugmat
    \let\vec=\spalignvector
    
%%%%%% row ops
    \newcommand\ro[2]{\xrightarrow[#2]{#1}}
%%%%%%%%%%%%%%%%%%%%%%%%%%%%%%%%%%%%%%%%%%%%%%%%%%%%%

%%%%%%%%%%%%%  PACKAGES (NOT INCLUDED IN CLASS) %%%%%%%%%%%%%%
\usepackage[delims={[]}]{spalign}

%%%%%%%%%%%%%%%% ALGORITHM TEMPLATE %%%%%%%%%%%%%%%%%%%%

%\begin{algorithm}[H]
%\SetAlgoLined\KwData{this text}
%\KwResult{how to write algorithm with \LaTeX2e }initialization\;
%\While{not at end of this document}
%	{read current\;\eIf{understand}{go to next section\;current section becomes this one\;}{go back to the beginning of current section\;}}
%	\caption{How to write algorithms}
%\end{algorithm}

%%%%%%%%%%%%%%%%%%%%%%%%%%%%%%%%%%%%%%%%%%%%%%%%%%%%%

\begin{document}

%%%%%%%%%%%%%%  DISCLAIMER  %%%%%%%%%%%%%%%%%%%%%

\begin{abstract}
	\par{These lecture notes were collated by me from a mixture of sources , the two main sources being the lecture notes provided by the lecturer and the 
content presented in-lecture. All other referenced material (if used) can be found in the \ita{Bibliography} and \ita{References} sections.}
	\par{The primary goal of these notes is to function as a succinct but comprehensive revision aid, hence if you came by them via a search engine , please note 
that they're not intended to be a reflection of the quality of the materials referenced or the content lectured.}
	\par{Lastly, with regards to formatting, the pdf doc was typeset in \LaTeX , using a modified version of Stefano Maggiolo's \href{http://blog.poormansmath.net/
latex-class-for-lecture-notes/}{\underline{\textcolor{blue}{class}}}}
\end{abstract}
\newpage

%%%%%%%%%%%%% LECTURES %%%%%%%%%%%%%%%%%%%%%%%

\section{Introduction}

\subsection{ILOs - Fundamentals}

\par{The goal of the course will be to deepen your understanding of the fundamental
principle and techniques in computer systems, such as:}
\begin{itemize}
	\item Memory and computation as fundamental resources of computing
	\item Representation of data structures in memory and the role of data types
	\item Techniques for management of computational resources
	\item Reasoning about concurrent systems
\end{itemize}


\subsection{Systems vs Application Software}

\par{We contrast system software with application software. System software can be seen as
low-level software, i.e it usually concerns itself with interacting with the machine directly or
very close to the metal whilst providing abstractions for application software}
\par{There are constraints which come with the nature of system software, such as fast
execution time, low memory consumption or low energy usage. Because high-level, managed languages are usually
highly abstract, it is almost impossible to build software which abides by these constraints, hence
one uses system programming languages like \texttt{C} which provide the programmer with more fine
grained control over how their program will execute}


\subsection{History}

\par{In the 1950s the distinction between these two types of software was nonexisting.
Machines were very much purposely built to run certain applications and a single executing
application would use the entire machine. \ita{Grace Hopper} was responsible for writing one of the
first compilers which turned human readable code into machine code.}

\par{Until the 1970s system software was always written in a processor specific assembly
language, which meant that for each new processor a systems programmer would need to rewrite the
same program in a different assembly language. In the 1970s \ita{Dennis Ritchie} and \ita{Ken Thompson} , while trying to port \texttt{UNIX} between two machines, invented \texttt{C} as a
\ita{portable, imperative} language which supported \ita{structured programming.}}

\par{In the 1980s, \ita{Stroustroup} came up with \texttt{C++} primarily so that he could
automate certain tasks which have to be done manually in C.}

\par{Until the 2010s the focus very much shifted towards the development of high-level, managed
languages however, the emergency of mobile devices rekindled the interested in new systems languages such as \texttt{Swift} and \texttt{Rust} which permit the devices to not waste their already scarce resources in tasks such as garbage collection}

\section{C - Introduction}

\key{introduction to C}{fundamental features}{differences}{strong
typin}{lexical scope}{lifetime of variables}{call by value}{declarations and definitions}{compiler
errors and warnings}{data representation in memory}{data types}

\subsection{A tale of two languages}

\par{In order to illustrate the advantages of C over managed languages take the following programs}

\noindent\begin{minipage}[t]{.45\textwidth}
	\begin{lstlisting}[language=python,]

	x = 41
	x = x + 1
	\end{lstlisting}
	\end{minipage}
\noindent\begin{minipage}[t]{.45\textwidth}
	\begin{lstlisting}[language=C,]
int main() {
	int x = 41;
	x = x + 1;
}
	\end{lstlisting}
\end{minipage}


\rem{Java syntax was heavily borrowed from C. Programs differ in how they are executed and on how memory is organised}

\par{The size (in memory) of \texttt{x} in Python is dependent on the architecture, in this computer for example is about 28B, whilst in C is just 4B. Why? Given the dynamic nature of python, its C implementation uses a descriptor object which stores the alongside the value which significantly increases the memory requirements. In C integers are usually 4B and that's all you need to represent \texttt{x} in memory }

\par{What about the number of instructions? That's even harder to know in python, because for example each operation must type-check first, then it must also check that addition is a valid operation and it represents all of that in some structure in the CPU. In C however, we know that only 3 instructions are required, 1 \texttt{add} and 2 \texttt{mov}.}

\rem{This level of fine-grained control allows us to confidently reason about the execution behaviour and performance of a program}

\subsection{Compiling}

\par{The compiling of code is the act of transforming the source code of a program into an executable file which can be run by the OS. There are two kinds of source files in C:}

\begin{itemize}
	\item \href{https://gcc.gnu.org/onlinedocs/cpp/Header-Files.html}{Header Files} : a file containing C declarations and macro definitions to be shared between several source files. Its contents can be requested in a program by including it, with the C preprocessing \textit{directive} \texttt{\#include}.  It is conventional to end file names with \texttt{.h}

	\item Compilation Units : These are \texttt{.c} files whose directives are replaced by the contents of the header files. Each file is compiled separately to keep compilation times short
\end{itemize}

\par{The compilation process can be split into 3/4 main steps:}

\begin{itemize}\item{Preprocessor : During the preprocessing stage the preprocessor will look at the source code and replace all directives and macros with the contents of their corresponding files. For examples \texttt{\#include <stdio.h>} will copy the contents of the I/O library \texttt{stdio.h}}

	\rem{To see an example of the output run \texttt{gcc -E}}

	
	\item{Compiler \& Assembler : The compiler transforms the preprocessed code into assembly code which in turn are turned into \textit{object} files \texttt{.o} (almost executable) by the assembler}

	\item{Linker : The linker will then find, for each name appearing in the object code, the address that was eventually assigned to that name, make the substitution, and produce a true single binary executable in which all names have been replaced by addresses}
\end{itemize}

\rem{You can use constants by using the \texttt{\#define} compiler directive which will essentially
replace every instance of the named variable with the actual value at preprocessing stage}


\subsubsection{Warnings \& Errors}

\par{Compiling errors mean that it is impossible to translate the program into an executable.
Warnings on the other end indicate that there is something and unusual in the code which will most
likely result in a runtime error. Occasionally one might want to keep the piece of code which threw the warning , in that case make sure to make that
clear in your code.\mymarginpar{For the coursework make sure that
to submit code which compiles without warnings}} 

\par{The compiler and the linker will throw different kinds of errors, being able to identify which is which helps with debugging. For example, it is possible to convert a program into object code which has a call to an undefined function this will just throw a warning. However, when running the linker an error will be thrown and an executable won't be created.}

\rem{\texttt{-Werror} turns all warnings into errors, the \texttt{-Wall} flag enables most compiler
warnings}








\section{The Scientific Method - Controls \& Sample Sizes in Experimental Design}

    \par{A quick introduction to the \ita{scientific method} and the \ita{philosophy of science} }

 	\begin{itemize}
 		\item Introduces high level of concepts of inference and how this can be used in the search for new knowledge.
 		\item Discusses the basics of the scientific method, hypotheses, and hypothesis testing. 
 		\item Finishes with a broad look at some methods often used in computing science research.
 	\end{itemize}

 	\subsection{Philosophy of Science}

 		\par{PS is a branch of philosophy that analyses methods of scientific inquiry. It challenges practices and assumptions which are taken for granted, and in doing so challenges the way we look at the world and gives us a way of reflecting on the scientific method and the knowledge generated so far.}

 		\par{Some philosophers do not believe however, that a simple criterion for identifying what Science is can ever be found. Karl Popper a prominent 20\textsuperscript{th} century philosopher attempted to suggest that scientific theories \ita{must} be falsifiable. For Popper, the line between science and pseudo-science is drawn when a theory cannot be shown to be incompatible with possible empirical observations, whether by virtue of it being tweaked so as to accommodate them (e.g. \ita{Marxism}) or by virtue of it being consistent with all possible observations (e.g. \ita{Psychoanalysis}.)
 		\par{Critics of Popper's theory point out that data can sometimes be at odds with theory, a contradicting observation does not necessarily falsifies a theory, and where to draw the line between a false theory or an aberrant observation is usually up to groups of scientists who work in the field. Furthermore, falsification can not help us decide between two competing but corroborated theories. To explore this in any more depth is far beyond the scope of this course, but the key takeaway, which Popper mentions in his response to its critics, is that:}

 		\quo{"the usefulness of falsifiability is that falsifiable conjectures say more, because they prohibit more and, in the case of their falsification, they lead to useful problems, which steer the creative process of science"}

 	\subsection{Scientific Inference}

 		\defn{Scientific Inference}{a conclusion reached on the basis of evidence and theories}

 		\defn{Deductive Inference}{the truth of its conclusion is \ita{guaranteed} by its premisses}

 		\defn{Inductive Inference}{the truth of its conclusion can not be solely determined by the truthfulness of its premisses}

 		\par{When reasoning by induction on the other hand, we draw conclusions about unobserved objects resorting to reasoning and \ita{past} observations of similar objects. Unlike in deduction true premisses might lead to false conclusions. Hume when discussing causation presents his famous illustration on how it is impossible to guarantee that a billiard ball will move when hit by another because to do so will mean that we assume the (more general)conclusion to be true right from the start, i.e. we believe that things that happen in the past in similar circumstances will happen again in the future \ita{because} in some other past, some other future \ita{was} indeed like that past. We are trying to reason about states of affairs in the world by appealing to logical necessity, but that the future \ita{must be} like the past is not necessarily true. The problem, according to Hume, lies in the fact that we conflate intuitions about our observed world with logical reasoning (\ita{relations of ideas).}

 		\par{Despite this problem, empirical sciences mostly use inductive inference when conducting research, this is because it is just not possible to observe every instance of our object of study. It is therefore important to be aware that certain truth is something that can not be achieved by induction and when conducting studies great importance must be taken when drawing conclusions about a given \ita{population} from a \ita{sample}. Note however that this does not mean that induction is not a valid method, and that empirical sciences have no legs to stand on, since though the problem of induction has not been solved there are strong critiques which are widely accepted. For example, \ita{Peter Strawson} argues that Hume himself is confused on what it means to reason about something. According to Strawson using induction is intrinsic to the act of reasoning}

 		\quo{The question whether induction is rational resembles ... the question whether the law is legal.}{Einstein}

 		\defn{Inference to the Best Explanation}{inductive reasoning which chooses the simplest hypothesis if several competing ones exist}

 		\par{As a compromise then, we can describe the goal of science to be that of drawing the \ita{most plausible} conclusion from observation where we define the most plausible explanation to be that which is \ita{as simple as possible, but no simpler}, said Einstein paraphrasing Occam.  

 		\quo{"entities should not be multiplied without necessity" Occam}{Occam}

 		\defn{Causal inference}{using induction to assert the causal connection between events}

 			\par{A common type of question that scientific research aims to answer is that of the source of causes of events and this is often achieved via inference. Hence, it still suffers from the problems discussed above, in particular here one has to be careful not to identify an event as a cause of another just because they are correlated. There might be a third unknown cause which causes both events to occur simultaneously.}
 			\par{In order to safely make causal inferences we use control groups in our experiments in order to isolate the factor being studied and control for the others.}

 			\defn{Control Group}{works as base line measure}

 			\defn{Treatment}{is the one where we should expect to see the effects of the \ita{"treatment"}}

 			\par{For example, if you wanted to study if small buttons lead to poor text entry then you would give devices with large buttons to the control and small to the treatment group. We would then compare the two groups, if we observe a significant difference between the groups, and given that both samples were taking from the same population, we can then be more inclined to infer causality. There can still be some \ita{confounding factors}, in order to help weed out those scientists often will also use randomised controls or more commonly in computer science, \ita{repeated-measures} where participants experience \ita{all} conditions}. In our example you could give the participants both type of devices}

 	\subsection{The Scientific Method}

 		\defn{Scientific Method}{a set of common principles, procedures and criteria for the objective investigation of the world}

 		$$\begin{array}{l}\text { General scientific method: } \\ \text { - Beliefs or observations about some something; } \\ \text { - Formulate a hypothesis to explain and predict that something; } \\ \text { - Test the hypothesis against empirical evidence; }\end{array}$$

 		\defn{Hypothesis}{a specific, falsifiable, parsimonious and predictive explanation for phenomena}

 		\par{Unlike Popper, Kuhn argues that during the period of what he calls \ita{normal science} data which contradicts an accepted hypothesis as a theory is initially seen as an error on the scientist's part. If enough of these build up then the current \ita{paradigm} reaches an existential crisis and that's when new hypothesis are proposed until eventually a new theory is accepted which subsumes parts of the older one. One should see the scientific method as iterative and the progress of science not as linear but instead brought by these sudden shifts in thought brought by dissonance within a field.}

 		\subsubsection{Experiments}

 		\defn{Experiment}{a series of steps used to gather data in against which we can test our hypotheses}

 		\par{An essential part of this process is the testing of hypotheses which is achieved by experiments carefully controlled by scientists\mymarginpar{we'll delve more into this in a future lecture}. Once we've gathered the data we use \ita{statistical analysis} to determine if the variance in data was caused by the experiment design or if occurred by chance.}

 		\rem{Do not take published \ita{"science"} at face value, always engage with the work. See the slides for examples of intentional and unintentional errors which can lead to \ita{"bad science"} and wrong conclusions}

 	\subsection{Science Dissemination}

 		\par{}

 	\subsection{Computing Science Research Method}
 		








%%%%%%%%%%%%%%  BIBLIOGRAPHY  %%%%%%%%%%%%%%%%%%%
\newpage
\nocite{*}
\printbibliography

%%%%%%%%%%%%%%%%%%%%%%%%%%%%%%%%%%%%%%%%%%

\end{document}
