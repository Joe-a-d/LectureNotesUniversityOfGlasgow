
\par{TOPICS: introduction to C, fundamental features, differences, strong
typing, lexical scope, lifetime of variables, call by value, declarations and definitions, compiler
errors and warnings,how data is represented in memory, why data types are fundamental in
C}

\section{Main}
\par{The main function is the entry point into the program. By default it returns 0 if the end
is reached when executing, any other non-negative return value indicates an unsuccessful execution}
\par{There are two valid versions of main, one of them is able to process command line
arguments} !!INSERT CODE!!

\section{printf}
\par{printf is a function defined in the stdio.h library, which takes a format string as its
first argument, and the following arguments as the values to be printed out}

!! INSERT LINK TO COMMON FORMAT STRINGS !!


\section{Variables}

\par{Variables are composed of a \ita{data type}, and \ita{identifier} and optionally an
\ita{initialisation expression}}

\rem{C is a \ita{statically typed} language, it is not possible to declare a variable without
declaring its type}

\subsection{Data Types}

\par{A \ita{bit-pattern} on it's own has no meaning, its meaning is context dependent. Data
types give bits meaning. By declaring a variable with a data type one decides what a bit-pattern
means when stored in memory}
\par{By enforcing operations to respect data types the compiler prevents meaningless
computations so that we preserve meaningful representation of the data. If \texttt{x} is a
\texttt{char} then \texttt{x + 1} has a very different meaning than if \texttt{x} was a float. The
resulting bit-patterns stored in memory would be totally different}

!! INSERT EXAMPLE FROM LECTURE SLIDES !!

\subsection{Representation in Memory}

\par{A variable is allocated a space in memory when it is declared and that location remains unchanged throughout
its lifetime, it is the data type of the variable which determine the bits being stored in memory.
It follows that for optimal performance one should take into consideration which data type to choose
depending on one's needs. For example, if one needs to store an array of small integers then one
could save almost 3MB of data by choosing to use \texttt{char} over \texttt{int}. Though in terms of
storage/tertiary memory this improvement is marginal, in terms of memory this can lead to a significant improve in
performance since primary memory provides faster access speeds at the cost of much lower capacity} 
\par{A lifetime of a variable depends on how the memory for that variable was allocated. There
are 3 possible cases:}
\begin{itemize}
	\item[Automatic] allocation happens for variables which are declared locally within a scope
	\item[Static] allocation means that the variable will persist throughout the execution of
	the program. Hence, every time that variable is needed the program will look at the same location
	\item[Allocated] allocation requires the programmer to manage the lifetime of the variable
	themselves by  explicitly request memory using \ita{dynamic memory function} such as \texttt{malloc}
\end{itemize}

\subsection{Stack-Based Memory Management}
\par{Automatically allocated variables' memory is automatically freed up once they expire. This
is achieved by following a LIFO \ita{stack-based memory management} approach. When the compiler sees a
start of the block it puts aside a location in memory for every variable in scope, when it reaches
the end of the block it frees up all previously assigned memory.}

\section{Other Data Types}

\subsection{Struct}

\par{Structs consist of a sequence of members which need not necessarily be all of the same
type. They are comparable to a public Java class without any methods, and similarly to Java a
struct's members can be accessed using dot notation. A major advantage of structs is that there is
no overhead in memory management since its variables are stored in memory sequentially in the order
in which they were declared.}
\par{One declares a struct type by using the keyword \texttt{struct} before the identifier, but
because one might want to initialise multiple structs with different values \texttt{typedef} can be
used to name that struct. After doing so, the struct keyword can be dropped}

\subsection{Arrays}
\par{Arrays consist of multiple elements of the same type. So that the memory is automatically
managed the size of an array stored in the stack  must have a fixed size \mymarginpar{Dynamic arrays
are possible if stored in the heap, see next week's lecture notes}}

\rem{array elements are stored next to each other in memory}

\rem{Arrays in C do not carry any information regarding their size. This means that it is not
possible to generate arrays dynamically, and one should always make sure to keep track of this
information}

\subsection{Strings and Characters}

\par{In C strings are just an array of characters, the compiler identifies the end of the array
by the \texttt{\\0} character which is appended to its end. This happens automatically if the string
is defines as a string literal \texttt{""} but must be added manually if the string is defined as an
array of chars \texttt{''}}

\subsection{Function}
\par{Function declarations represent its interface, i.e they let one know what is expected when
the function is called; the behaviour of the function can then be \ita{defined} inside a block. In C
all identifiers must be declared before they are used. To declare a identifier just means to
assign a name and type to that identifier, it allows the compiler to know the properties of that
identifier. Functions are no different, except that one must also identify the type of their
parameters. When defining a function one must give it a body specifying its behaviour}    

\rem{the linker is responsible for linking the reference of a function to its implementation}

\subsubsection{Call-by-value}
\par{When an argument is passed to a function its value is passed to the function's parameter,
hence changes made to the parameter have no effect on the argument. What happens is that a copy of
the argument's value is created and a new local variable whose scope is the function's body is
allocated some other space in memory; as it was mentioned above once the function ends execution
that location is automatically freed up, hence the changes will not persist unless the value is
returned}
\par{Arrays are special because when passed to a function they are not copied (expensive),
instead the address/pointer of its first element is passed\mymarginpar{remember your first attempt
at trying to write a function which returned an array's size}}
