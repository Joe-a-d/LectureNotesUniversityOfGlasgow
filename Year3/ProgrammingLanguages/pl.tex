
%% CLASS MANUAL FOUND IN http://blog.poormansmath.net/latex-class-for-lecture-notes/ %%
%% CLASS AUTHOR Stefano Maggiolo %%

%%%%%%%%%%%%%%%  TITLE PAGE  %%%%%%%%%%%%%%%%%%%
\documentclass[english,course]{Notes}
\title{Programming Languages}
\subject{SUBJECT AREA}
\author{Joao Almeida-Domingues}
\email{2334590D@student.gla.ac.uk}
\speaker{LECTURER}
\date{15}{01}{2021}
\dateend{19}{03}{2021}
\place{University of Glasgow}


%%%%%%%%%%%% BIB CONFIG %%%%%%%%%%%%%%%
\usepackage[backend=biber, style=reading]{biblatex} 

\bibliography{b.bib} %add bib file name

%%%%%%%%%%% LAYOUT  %%%%%%%%%%%%%%%

\renewcommand{\abstractname}{\vspace{3\baselineskip}} %hack to remove abstract


%%%%%%%%%%%%%%%%%%%%%%%%%%%%%%%%

%%%%%%%%%%%%%% KEEP HERE (conflict when in class) %%%%%%%%%%%%%%%%%%%%

 %%%%%MATRICES
    
    \let\mat=\spalignmat
    \let\amat=\spalignaugmat
    \let\vec=\spalignvector
    
%%%%%% row ops
    \newcommand\ro[2]{\xrightarrow[#2]{#1}}
%%%%%%%%%%%%%%%%%%%%%%%%%%%%%%%%%%%%%%%%%%%%%%%%%%%%%

%%%%%%%%%%%%%  PACKAGES (NOT INCLUDED IN CLASS) %%%%%%%%%%%%%%
\usepackage[delims={[]}]{spalign}

%%%%%%%%%%%%%%%% ALGORITHM TEMPLATE %%%%%%%%%%%%%%%%%%%%

%\begin{algorithm}[H]
%\SetAlgoLined\KwData{this text}
%\KwResult{how to write algorithm with \LaTeX2e }initialization\;
%\While{not at end of this document}
%	{read current\;\eIf{understand}{go to next section\;current section becomes this one\;}{go back to the beginning of current section\;}}
%	\caption{How to write algorithms}
%\end{algorithm}

%%%%%%%%%%%%%%%%%%%%%%%%%%%%%%%%%%%%%%%%%%%%%%%%%%%%%

\begin{document}

%%%%%%%%%%%%%%  DISCLAIMER  %%%%%%%%%%%%%%%%%%%%%

\begin{abstract}
	\par{These lecture notes were collated by me from a mixture of sources , the two main sources being the lecture notes provided by the lecturer and the 
content presented in-lecture. All other referenced material (if used) can be found in the \ita{Bibliography} and \ita{References} sections.}
	\par{The primary goal of these notes is to function as a succinct but comprehensive revision aid, hence if you came by them via a search engine , please note 
that they're not intended to be a reflection of the quality of the materials referenced or the content lectured.}
	\par{Lastly, with regards to formatting, the pdf doc was typeset in \LaTeX , using a modified version of Stefano Maggiolo's \href{http://blog.poormansmath.net/
latex-class-for-lecture-notes/}{\underline{\textcolor{blue}{class}}}}
\end{abstract}
\newpage

%%%%%%%%%%%%% LECTURES %%%%%%%%%%%%%%%%%%%%%%%


\section{Syntax}

\key{syntax}{specification}{BNF}{EBNF}{Regex}{terminal, nonterminal, sentence symbols}{production rule}{syntax tree}{grammar}{phrase}

\defn{Syntax}{concerns the \emph{form} of the program, i.e how constructs of
the PL (e.g expressions) need to be arranged to make a
\emph{well-formed} program}

\defn{Semantics}{has to do with the \emph{meaning} of the program, i.e its
behaviours when run}

\defn{Pragmatics}{how the language is to be used in practice}

\par{Syntax is important because it influences how programs are
\emph{written} by the programmer, \emph{read} by other programmers, and
\emph{parsed} by the computer. On the whole scheme of things though, the
other two are more important since they weigh more heavily on \emph{how}
a programmer goes about writing her program. Semantics determines how
programs are composed by the programmer, understood by other
programmers, and interpreted by the computer. Pragmatics influences how
programmers are expected to design and implement programs in practice.}

\mymarginpar{it's the content of the message which is important, not
the language in which it is written sort of thing}


\subsection{Specification}

\par{The syntax of a language must be specified , this can either be done
\emph{informally} or \emph{formally} the obvious trade-offs are between
accessibility and practicality vs precision and consistency}


\example{Informal}{

A while-command consists of \texttt{'while'}, followed by an expression enclosed
in parentheses, followed by a command.
}

\example{Formal}{

\begin{verbatim}
while-command = 'while' '(' expression ')' command
\end{verbatim}

}

\subsubsection{Formal Specification
Notations}

\begin{quote}
REs are good for specifying the syntax of lexical elements of programs
(e.g identifiers)
\end{quote}

\begin{quote}
Backus Naur Form (BNF) works well when specifying larger and nested
program constructs (e.g expressions, commands)
\end{quote}

\begin{quote}
Extended BNF (EBNF) combines RE and BNF
\end{quote}


\subsection{Grammar}


\defn{Grammar}{describes how to form strings from a language's alphabet that
are valid according to the language's syntax}


\defn{Context-Free Grammar}{grammar in which every \emph{production rule} is
of the form \(A \to \alpha\) where \(A\) is a single \emph{nonterminal}
symbol, and \(\alpha\) is a string of terminals and/or (\(\alpha\) can
be empty)}


\defn{Production Rule}{rules specifying a symbol substitution that can be
recursively performed to generate new symbol sequences .}

\example{}{
\(\langle\mathrm{Stmt}\rangle \rightarrow\langle\mathrm{Id}\rangle=\langle\operatorname{Expr}\rangle\)
replaces \(\langle\mathrm{Stmt}\rangle\) by
\(\langle\mathrm{Id}\rangle=\langle\operatorname{Expr}\rangle\)}


\defn{Terminal Symbols}{are literal symbols which occur in the language and may
show up in the output of the rules but cannot be changed by them, e.g
\texttt{a}}

\defn{Nonterminal symbols}{are symbols which can be replaced like
\(\langle\mathrm{Stmt}\rangle\)}


\defn{Sentence Symbol}{the nonterminal symbol that stands for a \emph{complete}
sentence}

\par{We use terminal and nonterminal symbols to specify the production rules.}


\subsection{BNF}

\par{BNF is a notation for expressing a grammar, where each element $\alpha , \beta, \lambda$ is a sequence of terminal and nonterminal symbols}

$$N = \alpha \ | \ \beta \ | \ \lambda$$

\example{mini-English \\}{

\texttt{subject\ =\ \textquotesingle{}I\textquotesingle{}\ \textbar{}\ \textquotesingle{}a\textquotesingle{}\ noun\ \textbar{}\ \textquotesingle{}the\textquotesingle{}\ noun}
  \ with
\texttt{\textquotesingle{}I\textquotesingle{},\ \textquotesingle{}a\textquotesingle{},\ \textquotesingle{}the\textquotesingle{}}
being terminal symbols and \texttt{noun} nonterminal

}

\example{Calc \\}{

\begin{itemize}
\item
  variables : \texttt{{[}a\ ...\ z{]}}
\item
  expressions made of variables , numerals and arithmetic operators
\item
  assignment and output commands
\item
  terminal :
  \texttt{‘put’\ ‘set’\ ‘=’\ ‘+’\ ‘-’\ ‘*’\ ‘(’\ ‘)’\ ‘\textbackslash{}n’\ ‘a’\ ‘b’\ ‘c’\ \ldots{}\ ‘z’\ ‘0’\ ‘1’\ \ldots{}\ ‘9’}
\item
  nonterminal : \texttt{prog\ expr\ num\ com\ prim\ id}
\item
  sentence : \texttt{prog} ( a sequence of 0+ \texttt{coms} followed by
  an \texttt{EOF})
\item
  production rules (two examples)
\end{itemize}

\mymarginpar{see lecture 1@28 for full spec}

\begin{verbatim}
expr = prim                         prim = num
     | expr `+` prim                     | id
     | expr `-` prim                     | '(' expr ')'
     | expr `*` prim
\end{verbatim}


}
\subsection{Phrase Structure \& Syntax
Trees}


\defn{Phrase Structure}{how phrases may be formed from sub-phrases}

Every phrase in the language has a \emph{syntax tree} that explicitly
represents its phrase structure. For a grammar \(G\) a syntax tree of
\(G\) is a tree with the following properties:

\begin{itemize}
\item
  Every terminal node is labeled by a terminal symbol of \(G\)
\item
  Every nonterminal node is labeled by a nonterminal symbol of \(G\)
\item
  A nonterminal node labeled \(N\) may have children labeled \(X,Y,Z\)
  only if \(G\) has a production rule \(N = X\ Y\ Z\) or
  \(N = \dots | X \ Y \ Z | \dots\)
\end{itemize}

\begin{figure}[H]
\centering
\includegraphics[width=\textwidth]{assets/syntax.png}
\end{figure}



%%%%%%%%%%%%%%  BIBLIOGRAPHY  %%%%%%%%%%%%%%%%%%%
\newpage
\nocite{*}
\printbibliography

%%%%%%%%%%%%%%%%%%%%%%%%%%%%%%%%%%%%%%%%%%

\end{document}
