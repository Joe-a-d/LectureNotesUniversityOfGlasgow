
%% CLASS MANUAL FOUND IN http://blog.poormansmath.net/latex-class-for-lecture-notes/ %%
%% CLASS AUTHOR Stefano Maggiolo %%
\documentclass[english,course]{Notes}

\title{MATHEMATICS 1R}
\subject{Mathematics}
\author{Joao Almeida-Domingues}
\email{2334590D@student.gla.ac.uk}
\speaker{David Palazzo \ita{(Calculus)} \\
Georgios Antoniou \ita{(Algebra)}}
\date{17}{09}{2018}
\dateend{30}{11}{2018}
\place{University of Glasgow}

 %%%%% GENERAL MATHEMATICAL NOTATION SHORTCUTS %%%%%
 
\newcommand{\n}{\mathbb{N}}
\newcommand{\z}{\mathbb{Z}}
\newcommand{\q}{\mathbb{Q}}
\newcommand{\cx}{\mathbb{C}}
\newcommand{\real}{\mathbb{R}}
\newcommand{\field}{\mathbb{F}}
\newcommand{\ita}[1]{\textit{#1}}
\newcommand{\oneton}{\{1,2,3,...,n\}}
\newcommand\ef{\ita{f} }
 %\qedhere to force QED in place
\newcommand\inv[1]{#1^{-1}}
\newcommand\setb[1]{\{#1\}}
\newcommand\en{\ita{n }}
\renewcommand\qedsymbol{QED}

%%%%%%%%%%%%%%%%PACKAGES%%%%%%%%%%%%%%%%%%%%%%%%%%%%%
\usepackage{lipsum}  
\usepackage{MnSymbol,amsmath,amsthm,amssymb,graphicx,mathtools,tikz,pgfplots} %maths
\usepackage{hyperref,framed,color,fancybox} %layout
% framed :  \begin{shaded,frame,snugshade or leftbar} \definecolor{shadecolor}{rgb}{XYZ} to change color
%fancybox: \shadowbox,ovalbox or doublebox
%%%%%%%%%%%%%%%%%%%%%%%%%%%

%%%CLASS SHORTCUTS%%%%
%\lecture{day}{month}{year} for margin note
%\begin{theorem} sdfsdf\end{theorem}
%\begin{proposition} dfsdfs\end{proposition}
%\begin{lemma} dsfsd \end{lemma}
%\begin{corollary} f ffew \end{corollary}
%\begin{definition} fwewef w \end{definition}
%\begin{example} feww e\end{example}
%\begin{exercise} wefwe \end{exercise}
%\begin{remark} wef we \end{remark}
%\begin{fact} wefe \end{fact}
%\begin{problem} wef ew \end{problem}
%\begin{conjecture} ewfew \end{conjecture}
%\begin{claim} few w \end{claim}
%\begin{notation} fewf \end{notation}
%%%%%%%%%%%%%%%%%%%%%%%%%

%BIBLIO: mathworld.wolfram.com@Linear Algebra A modern Introduction 
\begin{document}
\newpage
\section{Properties of Numbers}

\subsection{Introduction}

\defn{A set}{ is a finite or infinite collection of objects in which order has no significance, and multiplicity is generally also ignored}

\nota{ $ A = \{ a_1 , a_2, a_3 \}$}

\nota{$a \in A$ , The element \ita{a} belongs to the set \ita{A}}
\nota{$ A \subseteq B$, \ita{A} is a subset of \ita{B}, all elements of \ita{A} appear in \ita{B}  but the opposite is not necessarily true}
\nota{$ A \subset B$, the set \ita{B} has at least one element which does not belong to \ita{A}}
\nota{$a \notin A , A \nsubseteq B , A \not\subset B $, are negations of the above}

\defn{Natural Numbers}{ $\n$, the set of all positive whole numbers }
\defn{Whole Numbers}{ $\n_0 \ ,\  \n + 0$}
\defn{Integers}{ $\z \ ,\ \ \n_0 + $ negative whole numbers}
\defn{Rational Numbers }{$\q$, the set of all fractions with  denom $\neq$ 0}
\defn{Irrational Numbers}{ $\real \setminus \q$, the set of numbers which can not be expressed as a fraction}
\defn{Real Numbers}{ $\real$, the set of all numbers in the real line}

\rem{ $\n \subset \n_0 \subset \z \subset \q \subset \real$ and $ \real \setminus \q \subset \real $}

\ex{Prove that $\sqrt{n} \in \real \setminus \q \ , \ n > 1$ for any squarefree integer n, e.g. $\sqrt{20}$}

\textbf{Method}
\begin{enumerate}
	\item Proof that any $j \cdot g \in \real - \q , \text{ with } j,g \in \q \ , \ \real - \q$
	\item Decompose $\sqrt{(n)} \text{ into } k\sqrt{(u)}$, where u is prime 
	\item Prove that $\sqrt{(u)}$ is irrational by Euclid's Lemma Method
\end{enumerate}

\proofs{ \hphantom{.}
\begin{itemize}
	\item[  Step 1]
	
	Let $\frac{a}{b} , \frac{c}{d} \in \q \text{ and } x \in \real - \q$:
	
	\begin{alignat}{2} 
	& x \cdot \frac{a}{b} &&= \frac{c}{b} \\ 
	 \iff & x &&= \frac{cb}{da} \label{eq:cont1}
	\end{alignat}
	
We reach a contradiction at \eqref{eq:cont1} since integers are closed under addition, i.e. $x$ must belong to $\q$  \mymarginpar{A set has closure under an operation if performance of that operation on members of the set always produces a member of the same set}
\end{itemize}
}

 \begin{equation} \sqrt{20} = 2\sqrt{5} \tag{Step 2} \end{equation}

\proofs{
\begin{itemize}
\item[  Step 3]
It follows from \ref{Step 2} that $2\sqrt{5}$ is rational \ita{iff} 2 and $\sqrt{5}$ are both rational.


	\item[Case 1:] 2 is rational is trivially true 
	\item[Case 2:]

Let's assume that $\sqrt{5} \in \q$. This assumption imples that 
$$ \exists \ a, b \in \z , b \neq 0 | \sqrt{5} = \sfrac{a}{b}$$

We can assume also that $\sfrac{a}{b}$ is in its most reduced form, i.e. it has no common factors. Then,

\begin{gather}
\sqrt{5}^2  = \frac{a^2}{b^2} \\
5b^2  = a^2
\end{gather}\label{eq:irr1}

\lem{\textbf{Euclid's} : For a prime number $p$ if $p \divides mn$ then $p \divides m$ or $ p \divides n$\label{euclid}}

It follows from \ref{eq:irr1} that 5 divides $a^2$. And from \ref{euclid} (by setting $p=5 , m,n=a$) that 5 divides $a$. \\
We then have that $a = 5k , k \in \z$, and can therefore rewrite \ref{eq:irr1} :

\begin{align}
5b^2 &= (5k)^2 \\
b^2 &= 5k^2
\end{align}\label{eq:irr1}

Similarly as above it follows then, that 5 divides $b$. Hence we reach the conclusion that 5 divides both $a$ and $b$. Which contradicts our original assumption that $ a \text{ and } b $ are coprime i.e. that $\frac{a}{b}$ was in its most reduced form.  \\

$\therefore \sqrt{5}$ is not irrational $\implies \sqrt{20}$ is not irrational


\end{itemize}

}

\lem{\textbf{Division Algorithm} Given two natural numbers $a \text{ and } b$, there integers $q \text{ and } r$ , such that: $$ a = bq + r \ \ \ q \geq 0 \text{ and } 0 \leq r < b $$} \label{DivAlg}

\proofs{ Let $ \frac{a}{b}$ be the positive natural number between $q$ and $q+1$. 
\begin{align*}
 & q \leq \frac{a}{b}  < q + 1 \\
 & bq \leq a < bq + b \\
 & 0 \leq a - bq < b \\
\end{align*}

Setting $ r = a - bq$, we obtain $ a = bq+r , q \geq 0 \text{ and } 0 \leq r < b$
}
\lem{\textbf{Euclidean Algorithm} is obtained by repeated application of \ref{DivAlg}. 
$$ \text{gcd}(a,b) = am + bn$$}
 
 \proofs{
 
 \begin{enumerate}
 \item gcd $(a,b) =$ gcd$(b,r) = $ ... $=$ gcd$(z,0)$ \\ \\
 Why is this true? \\
 If $d | ab, \text{then there exists a } k, l \text{ s.t. } a = dk \text{ and } b = dl$ \ref{DivAlg} also tells us that $a = bq + r $, replacing $dk$ and $dl$ , we obtain the following: 
 \begin{align*}
 	& dk = dlq + r \\
	& r = d(k - lq)
 \end{align*}
 
 Hence, there exists a number (represented above by $(k-lq)$) that multiplied by $d$ gives us $r$. Therefore, by definition, $d$ must also divide $r$
 
 \item Assuming that $ a > b > r$ , the sequence will decrease until eventually the remainder can not be further divided, i.e. $r = 0$. Hence,
 
 $$ \text{gcd}(a,b) = \text{gcd}(b,r) = \text{gcd}(r,0) $$
 
By looking at the last term of the equality we find that the gcd must by definition be r (since it divides exactly). Therefore, by back-substituting the result \footnote{recursively, haskell style} we find that the \textbf{gcd of the original expression is equal to the remainder of the penultimate}
 \end{enumerate}}
 
 \newpage
 \ex{ Find integers m and n such that $55m + 7n  = 1$. Note that by the definition of the division algorithm, if such numbers exist, then we expect gcd$(55,7) = 1$. So, we start by applying the division algorithm to verify that it exists:
 
 \begin{align*}
 55 &= 7 \cdot 7 + 6 \\
 7 &= 6 \cdot 1 + \pmb{1} \\
 6 &= 1 \dot 6 + 0
 \end{align*}
 
\noindent We can now rearrange the equations above in the form $ r = a - bq$, and backtrack until we reach the original equation in terms of 55 and 7.
 
 \begin{alignat*}{2}
& 6  &&= 55 - 7 \cdot 7 \\
& 1 &&= 7 - 6 \cdot 1 \\
&  &&= 7 - (55 - 7 \cdot 7)(1) \\
& &&= 7(8) - 55
 \end{alignat*}
 
 \noindent Hence, $m = -1 \text{ and } n = 8$
 }

\defn{Polynomial}{ of degree $n \geq 0$ is an expression of the form:
	$$ a_nx^n + a_{n-1}x^{n-1} + \dots + a_1x + a_0$$
	
Where $a_n$ is the leading coefficient}
\rem{If $a_n = 1$, then the polynomial is said to be monic}
\rem{0 is considered to be a polynomial}

\rem{The Euclidean Algorithm can also be applied to polynomials, since: $$ f(x) = g(x)q(x) + r(x)$$ Where $0 \leq r(x) < g(x)$}

\theorem{\textbf{Remainder:} Let $f(x)$ be a polynomial in $x$, and $c$ be a constant: $$ f(x) = (x-c) q(x) + f(c) $$}

\theorem{\textbf{Factor:} If $f(c) = 0$ , then $(x-c)$ is a factor of $f(x)$}

\newpage 
\ex{Find the highest monic quadratic polynomial which divides both $$ g(x) = x^3 + 6x^2 + 11x + 6 \text{   and   } f(x) = x^4 + 5x^3 + 10x^2 + 20x + 24$$

Following a similar procedure as before, we can apply the Euclidean Algorithm until we find a polynomial which divides $g(x)$ and $f(x)$ exactly (i.e. $r(x) = 0$). The key point is to keep  in mind the fact that \textbf{any polynomial wich divides both $f(x)$ and $g(x)$ must also divide their remainder}.

\begin{enumerate}
\item Dividing $f(x)$ by $g(x)$ , so as to find $q(x)$ and $r(x)$:

$$\polylongdiv{x^4 + 5x^3 + 10x^2 + 20x + 24}{x^3 + 6x^2 + 11x + 6}$$

Hence, we have that: $$ f(x) = g(x)(x-1) + 5(x^2 + 5x + 6)$$

\item Continuing the algorithm, now with $g(x)$ and monic $r(x)$ obtained above:

$$\polylongdiv{x^3 + 6x^2 + 11x + 6}{x^2 + 5x + 6}$$

Since the remainder is 0, we find that $(x^2 + 5x + 6)$ divides exactly into $g(x)$, and therefore also divides into $f(x)$. Lastly, by looking at the remainder of the penultimate iteration, we can assert that, $x^2 + 5x + 6$ is the highest monic polynomial which divides both $f(x)$ and $g(x)$

\end{enumerate}
}

\newpage
\section{Complex Numbers}

\end{document}