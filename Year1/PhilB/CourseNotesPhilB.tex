
%% CLASS MANUAL FOUND IN http://blog.poormansmath.net/latex-class-for-lecture-notes/ %%
%% CLASS AUTHOR Stefano Maggiolo %%
\documentclass[english,course]{Notes}

\title{PHILOSOPHY 1B: HOW SHOULD I LIVE?}
\subject{Philosophy}
\author{Joao Almeida-Domingues}
\email{2334590D@student.gla.ac.uk}
\speaker{Jennifer Corns}
\date{07}{01}{2019}
\dateend{24}{05}{2019}
\place{University of Glasgow}

\usepackage[backend=biber, style=reading]{biblatex}
\bibliography{readings}
\usepackage{csquotes}

 %%%%% GENERAL MATHEMATICAL NOTATION SHORTCUTS %%%%%
 
\newcommand{\n}{\mathbb{N}}
\newcommand{\z}{\mathbb{Z}}
\newcommand{\q}{\mathbb{Q}}
\newcommand{\cx}{\mathbb{C}}
\newcommand{\real}{\mathbb{R}}
\newcommand{\field}{\mathbb{F}}
\newcommand{\ita}[1]{\textit{#1}}
\newcommand{\oneton}{\{1,2,3,...,n\}}
\newcommand\ef{\ita{f} }
 %\qedhere to force QED in place
\newcommand\inv[1]{#1^{-1}}
\newcommand\setb[1]{\{#1\}}
\newcommand\en{\ita{n }}
\renewcommand\qedsymbol{QED}
\newcommand\readings{\textbf{Readings:} \\}
\newcommand\sep{\\ \noindent\rule{10cm}{0.8pt} \\}
\newcommand\quo[1]{\begin{displayquote}\ita{\large{#1}}\end{displayquote}}

%%%%%%%%%%%%%%%%PACKAGES%%%%%%%%%%%%%%%%%%%%%%%%%%%%%
\usepackage{lipsum}  
\usepackage{amsmath,amsthm,amssymb,graphicx,mathtools,tikz,pgfplots} %maths
\usepackage{hyperref,framed,color,fancybox} %layout
% framed :  \begin{shaded,frame,snugshade or leftbar} \definecolor{shadecolor}{rgb}{XYZ} to change color
%fancybox: \shadowbox,ovalbox or doublebox
%%%%%%%%%%%%%%%%%%%%%%%%%%%

%%%CLASS SHORTCUTS%%%%
%\lecture{day}{month}{year} for margin note
%\begin{theorem} sdfsdf\end{theorem}
%\begin{proposition} dfsdfs\end{proposition}
%\begin{lemma} dsfsd \end{lemma}
%\begin{corollary} f ffew \end{corollary}
%\begin{definition} fwewef w \end{definition}
%\begin{example} feww e\end{example}
%\begin{exercise} wefwe \end{exercise}
%\begin{remark} wef we \end{remark}
%\begin{fact} wefe \end{fact}
%\begin{problem} wef ew \end{problem}
%\begin{conjecture} ewfew \end{conjecture}
%\begin{claim} few w \end{claim}
%\begin{notation} fewf \end{notation}

\begin{document}
\newpage

\section{Why be good?}

\subsection{Don't be good! Just appear to be}
\lecture{10}{01}{2019}

\readings \cite{Nielsen1984} \\ \cite{PlatoRepublicI}
\sep

\par{ In trying to answer the question "Why be good?", we start by analysing the socratic dialogue in \ita{Plato's - The Republic, Book I}. The main issue raised , which is left unanswered by the end of this text,  can be put forward as \ita{"Must we really be moral, can't we just fake it?"}

\par{ \mymarginpar{The Republic's analysis}The text opens with Socrates being invited to discuss philosophical matters with a well-to-do elder - Cephalus. For Cephalus, to be good is nothing but to be just which in term means simply , that all one needs to do is to follow simple rules, and is most definitely nothing to worry much about. According to him and his son, one follows the rules, makes some money, pays his debts and that's all there is to it. The rest of the dialogue is pretty much an attempt , on Socrates' part,  to show them that things are not quite as straightforward as they seem to think. To awaken them from their \ita{moral complacency}. Every time Cephalus elaborates in more detail what these rules of his are, Socrates comes up with a counter example which shows that to follow the rule seems to involve some kind of injustice, and that to break it is necessary if good is to be carried out. One particularly good one is when Socrates attacks the necessity of harming one's enemy. Both men agree that to cause harm is to cause the one who is harmed \ita{to deteriorate in excellence}, and since they had agreed at the start that justice is one of the human excellences/virtues, he enquires: how is it that by virtue of doing good, one can make others becomes less good?}

\quo{Can good men by the practice of their virtue make men bad?}

\par{After a while Thrasymacus gets fed up with Socrates' \ita{shtick}, to him Socrates is just another "negative nelly" who goes around town asking questions, destroying people's core beliefs, but offering no alternative. This leads to Socrates (pseudo-)humbly prompting Thrasymacus for his opinion. How does he define justice? Thrasymacus argument is simple. He agrees with the main argument of the first discussion, that people recognise justice by virtue of following certain rules (moral \& legal) . These rules are created by powerful individuals, which are usually taken in high-esteem, or at least are so powerful that their rules remain unchallenged. It only makes sense, Thrasymacus argues, that these rule-makers would make them so that they would be advantageous to them. Again Socrates challenges most definitions used by Thrasymacus and in a similar vain to above he asks: if men are fallible, the laws can be wrong, is it then right to follow them? Thrasymacus agrees that it is not, but he counters by challenging not the action of those who follow them, but the notion of leader itself. A leader who fails, only does so when \ita{"the knwoledge of the craft leaves them"}, and at that moment she is not a leader. He makes a stronger case for himself by appealing to empirical data, or common sense. As if saying; look around,  can't you see that the unjust, the crooks , the ruler of states, regardless of their political believes, wether they be tyrants or democrats, they are the ones with plenty of money and in good health who get to enjoy the pleasures of life. \ita{Book I} ends, with no answer for the issue raised by Thrasymacus, but both Plato and (Plato's) Socrates most definitely do not agree with this view, the whole of \ita{Book I} is an attempt on the part of Plato to show that there are some real questions that one must ask regarding what it means to leave "a good life", it is not by being complacent or a skeptic that one can find out the answers, but they do exist. Exposing them is his goal for the other Books.\mymarginpar{\rem{*Link with the nietzschean view of slave-master morality?}}

\quo{Those who reproach injustice do so because they are afraid not of doing it but of suffering it*.}

\par{Nielsen agrees that nihilism and ethical scepticism can not be so easily dismissed. He disagrees with the arguments based on the view that to even ask "Why be good?" is non-sensical. His opponents argue that \ita{to say that an act is right just means that one should do it}. He gives enough lead to this argument, he even advances that the ammoral one can very well understand this, from the moral point of view, he claims, \ita{"I should do what everything considered, I regard as right"} can be taken as an analytic truth. But that is not what the immoral agent is asking, he's instead asking "Why should I take the moral point of view at all" , why must moral reasons be taken as overriding reasons for action? Just like Thrasymacus argues above, why must one be compelled by morals and not self-interest? One could argue from an Hobbesian point of view, that moral institutions and the implicit social contract are a necessity if well-being among men is to flourish.\mymarginpar{Link with Nielsen}   But this only shows that \ita{collectively} moral institutions are necessary. But we still hit against the fact discussed above, that those who make these moral institutions can still make them in their own light, with self-interested reasons in mind. Or taking it further, why not just simulate these moral behaviours when in polite company, so as "not to make waves"? Nielsen believes this to be an honest possibility and that it is not unreasonable to leave such a live. In today's age is widely accepted that the less pain and suffering there is, the better, but happiness is not necessarily linked with morality. To be unhappy because one is immoral is contingent and not necessarily true. 

\quo{Will any person, no matter how he is placed in society, or what kind of society he is in, be unhappy or at least less happy, if he is an unprincipled bastard?}

Even if one takes the view that this person would be so awful, that she would be unable to enjoy the pleasures of life that come with the fact that humans are a social animal. This does not mean that this person could not be a sort of \ita{classist amoralist}, as Nielsen puts it. He takes Thrasymacus' arguments further, and shows that if one is powerful enough, one can be a (sophisticaded) tyrant for many (where one treats the dominated classes as mere means to one's ends) while still extending his interest to his peers (where one's able to develop genuine relationships). This is the key distinction between Thrasymacus' all powerful tyrants as \ita{ethical egoists}, and Nielsen's more polished \ita{classist amoralists}. The latter asks "Why not be morally arbitrary?" if it has be shown that "the peace" can still be kept? Nielsen argues that this can not be shown by pure reason, and that the burden of proof lies with those who argue , from along Aristotelian lines, that classical immoralism rests in a mistake, in a lack of knowledge.


\newpage
\subsection{To avoid punishment}
\lecture{14}{01}{2019}

\readings \cite{PlatoRepublicII}

\subsection{Because it's all you can do!}
\lecture{15}{01}{2019}

\readings \cite{PsychEgoism} \\ \cite{EthicalEgoism}

\subsection{Because it's good for you to be good! }
\lecture{16}{01}{2019}

\readings \cite{PlatoRepublicIX} \\ \cite{PlatoRepublic2} \\ \cite{AristotleI}

\subsection{Because of the sort of being that you are! }
\lecture{21}{01}{2019}

\readings \cite{HypotheticalImperatives}

\subsection{Because (duh) it's good! }
\lecture{23}{01}{2019}

\readings \cite{Protagoras} \\ \cite{AristotleVII}

\newpage
\printbibliography



\end{document}