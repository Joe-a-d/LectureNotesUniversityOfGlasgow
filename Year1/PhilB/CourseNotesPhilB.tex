
%% CLASS MANUAL FOUND IN http://blog.poormansmath.net/latex-class-for-lecture-notes/ %%
%% CLASS AUTHOR Stefano Maggiolo %%
\documentclass[english,course]{Notes}

\title{PHILOSOPHY 1B: HOW SHOULD I LIVE?}
\subject{Philosophy}
\author{Joao Almeida-Domingues}
\email{2334590D@student.gla.ac.uk}
\speaker{Jennifer Corns}
\date{07}{01}{2019}
\dateend{24}{05}{2019}
\place{University of Glasgow}

\usepackage[backend=biber, style=reading]{biblatex}
\bibliography{readings}
\usepackage{csquotes}

 %%%%% GENERAL MATHEMATICAL NOTATION SHORTCUTS %%%%%
 
\newcommand{\n}{\mathbb{N}}
\newcommand{\z}{\mathbb{Z}}
\newcommand{\q}{\mathbb{Q}}
\newcommand{\cx}{\mathbb{C}}
\newcommand{\real}{\mathbb{R}}
\newcommand{\field}{\mathbb{F}}
\newcommand{\ita}[1]{\textit{#1}}
\newcommand{\oneton}{\{1,2,3,...,n\}}
\newcommand\ef{\ita{f} }
 %\qedhere to force QED in place
\newcommand\inv[1]{#1^{-1}}
\newcommand\setb[1]{\{#1\}}
\newcommand\en{\ita{n }}
\renewcommand\qedsymbol{QED}
\newcommand\readings{\textbf{Readings:} \\}
\newcommand\sep{\\ \noindent\rule{10cm}{0.8pt} \\}
\newcommand\quo[1]{\begin{displayquote}\ita{\large{#1}}\end{displayquote}}

%%%%%%%%%%%%%%%%PACKAGES%%%%%%%%%%%%%%%%%%%%%%%%%%%%%
\usepackage{lipsum}  
\usepackage{amsmath,amsthm,amssymb,graphicx,mathtools,tikz,pgfplots} %maths
\usepackage{hyperref,framed,color,fancybox} %layout
% framed :  \begin{shaded,frame,snugshade or leftbar} \definecolor{shadecolor}{rgb}{XYZ} to change color
%fancybox: \shadowbox,ovalbox or doublebox
%%%%%%%%%%%%%%%%%%%%%%%%%%%

%%%CLASS SHORTCUTS%%%%
%\lecture{day}{month}{year} for margin note
%\begin{theorem} sdfsdf\end{theorem}
%\begin{proposition} dfsdfs\end{proposition}
%\begin{lemma} dsfsd \end{lemma}
%\begin{corollary} f ffew \end{corollary}
%\begin{definition} fwewef w \end{definition}
%\begin{example} feww e\end{example}
%\begin{exercise} wefwe \end{exercise}
%\begin{remark} wef we \end{remark}
%\begin{fact} wefe \end{fact}
%\begin{problem} wef ew \end{problem}
%\begin{conjecture} ewfew \end{conjecture}
%\begin{claim} few w \end{claim}
%\begin{notation} fewf \end{notation}

\begin{document}
\newpage

\section{Why be good?}

\subsection{Don't be good! Just appear to be}
\lecture{10}{01}{2019}

\readings \cite{Nielsen1984} \\ \cite{PlatoRepublicI}
\sep

\par{ In trying to answer the question "Why be good?", we start by analysing the socratic dialogue in \ita{Plato's - The Republic, Book I}. The main issue raised , which is left unanswered by the end of this text,  can be put forward as \ita{"Must we really be moral, can't we just fake it?"}}

\par{ \mymarginpar{The Republic's analysis}The text opens with Socrates being invited to discuss philosophical matters with a well-to-do elder - Cephalus. For Cephalus, to be good is nothing but to be just which in term means simply , that all one needs to do is to follow simple rules, and is most definitely nothing to worry much about. According to him and his son, one follows the rules, makes some money, pays his debts and that's all there is to it. The rest of the dialogue is pretty much an attempt , on Socrates' part,  to show them that things are not quite as straightforward as they seem to think. To awaken them from their \ita{moral complacency}. Every time Cephalus elaborates in more detail what these rules of his are, Socrates comes up with a counter example which shows that to follow the rule seems to involve some kind of injustice, and that to break it is necessary if good is to be carried out. One particularly good one is when Socrates attacks the necessity of harming one's enemy. Both men agree that to cause harm is to cause the one who is harmed \ita{to deteriorate in excellence}, and since they had agreed at the start that justice is one of the human excellences/virtues, he enquires: how is it that by virtue of doing good, one can make others becomes less good?}

\quo{Can good men by the practice of their virtue make men bad?}

\par{After a while Thrasymacus gets fed up with Socrates' \ita{shtick}, to him Socrates is just another "negative nelly" who goes around town asking questions, destroying people's core beliefs, but offering no alternative. This leads to Socrates (pseudo-)humbly prompting Thrasymacus for his opinion. How does he define justice? Thrasymacus argument is simple. He agrees with the main argument of the first discussion, that people recognise justice by virtue of following certain rules (moral \& legal) . These rules are created by powerful individuals, which are usually taken in high-esteem, or at least are so powerful that their rules remain unchallenged. It only makes sense, Thrasymacus argues, that these rule-makers would make them so that they would be advantageous to them. Again Socrates challenges most definitions used by Thrasymacus and in a similar vain to above he asks: if men are fallible, the laws can be wrong, is it then right to follow them? Thrasymacus agrees that it is not, but he counters by challenging not the action of those who follow them, but the notion of leader itself. A leader who fails, only does so when \ita{"the knwoledge of the craft leaves them"}, and at that moment she is not a leader. He makes a stronger case for himself by appealing to empirical data, or common sense. As if saying; look around,  can't you see that the unjust, the crooks , the ruler of states, regardless of their political believes, wether they be tyrants or democrats, they are the ones with plenty of money and in good health who get to enjoy the pleasures of life. \ita{Book I} ends, with no answer for the issue raised by Thrasymacus, but both Plato and (Plato's) Socrates most definitely do not agree with this view, the whole of \ita{Book I} is an attempt on the part of Plato to show that there are some real questions that one must ask regarding what it means to leave "a good life", it is not by being complacent or a skeptic that one can find out the answers, but they do exist. Exposing them is his goal for the other Books.\mymarginpar{\rem{*Link with the nietzschean view of slave-master morality?}}}

\quo{Those who reproach injustice do so because they are afraid not of doing it but of suffering it*.}

\par{Nielsen agrees that nihilism and ethical scepticism can not be so easily dismissed. He disagrees with the arguments based on the view that to even ask "Why be good?" is non-sensical. His opponents argue that \ita{to say that an act is right just means that one should do it}. He gives enough lead to this argument, he even advances that the ammoral one can very well understand this, from the moral point of view, he claims, \ita{"I should do what everything considered, I regard as right"} can be taken as an analytic truth. But that is not what the immoral agent is asking, he's instead asking "Why should I take the moral point of view at all" , why must moral reasons be taken as overriding reasons for action? Just like Thrasymacus argues above, why must one be compelled by morals and not self-interest? One could argue from an Hobbesian point of view, that moral institutions and the implicit social contract are a necessity if well-being among men is to flourish.\mymarginpar{Link with Nielsen}   But this only shows that \ita{collectively} moral institutions are necessary. But we still hit against the fact discussed above, that those who make these moral institutions can still make them in their own light, with self-interested reasons in mind. Or taking it further, why not just simulate these moral behaviours when in polite company, so as "not to make waves"? Nielsen believes this to be an honest possibility and that it is not unreasonable to leave such a live. In today's age is widely accepted that the less pain and suffering there is, the better, but happiness is not necessarily linked with morality. To be unhappy because one is immoral is contingent and not necessarily true.}

\quo{Will any person, no matter how he is placed in society, or what kind of society he is in, be unhappy or at least less happy, if he is an unprincipled bastard?}

Even if one takes the view that this person would be so awful, that she would be unable to enjoy the pleasures of life that come with the fact that humans are a social animal. This does not mean that this person could not be a sort of \ita{classist amoralist}, as Nielsen puts it. He takes Thrasymacus' arguments further, and shows that if one is powerful enough, one can be a (sophisticaded) tyrant for many (where one treats the dominated classes as mere means to one's ends) while still extending his interest to his peers (where one's able to develop genuine relationships). This is the key distinction between Thrasymacus' all powerful tyrants as \ita{ethical egoists}, and Nielsen's more polished \ita{classist amoralists}. The latter asks "Why not be morally arbitrary?" if it has be shown that "the peace" can still be kept? Nielsen argues that this can not be shown by pure reason, and that the burden of proof lies with those who argue , along Aristotelian lines, that classical immoralism rests in a mistake, in a lack of knowledge.


\newpage
\subsection{To avoid punishment}
\readings \cite{PlatoRepublicII}
\sep
\lecture{14}{01}{2019}

\par{In the start of Book II, Glaucon proposes to Socrates  to strengthen Thrasymachus argument, so that Socrates can argue against it , praising just by itself, and finally show that it is better to be just than unjust. Glaucon first expounds the common view of justice~(\ref{G:what}), he then argues that people who practice it only do so out of necessity, not for its own sake~(\ref{G:selfInterest}). Finally, he concludes that it is not only true that people act unjustly, but they do so because they have strong reasons to do it~(\ref{G:unjustBetter})}

\begin{enumerate}

	\item{What Justice Is?~\label{G:what}}
		\par{According to Glaucon, the common view amongst the people is that to do injustices is naturally good and to suffer them is bad. Hence they \ita{"make laws and covenants, and what the law commands they call lawful and just"}}
		
	\item{To Be Just is a Need only of the Weak~\label{G:selfInterest}}
		\par{It is actually so much worse to suffer them, that the only reason people practice good deeds is simply because they are not powerful enough to enact themselves.To be sufficiently powerful to escape the negative consequences brought upon by the doing of unjust deeds, and not to do them would be seen as stupid by everyone~\mymarginpar{All powerful ring story}}
	 
	\item{The Life of the Unjust is Better~\label{G:unjustBetter}}
		\par{Glaucon then asks Socrates to consider two diametrically opposed characters. One is the most skilled unjust person, who every attempt at injustice remains undetected and is therefore seen as a perfectly just man. The other a sort of martyr; a completely just man who \ita{"doesn't want to be believed to be good but to be so"}. Even though he does no injustice, his fellow men see him as the epitome of it. He again stresses that only a deluded one would prefer the latter's life. That the unjust man's life is better, is not only common sense, but is a matter of fact.
		
		\quo{When fathers speak to their sons, they say that one must be just, as do all the others who have charge of anyone. But they don't praise justice, itself, only the high reputations it leads to and the consequences of \textbf{being thought as just}}}
	
\end{enumerate}

\par{We've addressed this issues in Lecture1 above, it seems that Glaucon is advocating that to appear to be just is enough to avoid punishment and reap the benefits bestowed upon "the moral one". But could there be a reason to \ita{actually be} moral? Adeimantus claims that there are those who argue for this by appealing to \ita{divine punishment}}

\par{Until now we've focused on \ita{social punishment}, and we've raised several arguments that seem to fail to address the distinction between \ita{appearance of being} and \ita{actually being} moral. Given the view of the Gods as omnipresent however, it seems that this appearance/reality gap seems to disappear. In this view, some argue that the Gods give the just the goos they deserve and \ita{"bury the impious and unjust in mud in Hades"}. There are those others however, (in particular the poets) who argue that this doesn't seem to be the case. The Gods seem to like to be entertained, and enjoy human offerings (sacrifices). The unjust man is in a better position to offer such things, hence it seems that as long as he is successful enough, he can buy his eternal redemption. So as long as one believes in divine punishment and this portrayal of the Gods, the unjust is still on the lead}

\par{Socrates defends that justice is a good of the highest kind, good not only for its consequences, but also good in itself. Adeimantus fails to see these, he argues that if justice was like one of those goods (like seeing, hearing, knowing, being healthy...) then one would pursue them for its own sake and we would't need laws and deterrents. Can it be shown that morality is inherent to human rationality, and not just a way of pursuing our own self-interest?~\mymarginpar{Nielsen addressed this concern, see above. Plato won't fully address it until the end of Book IX}}

\newpage


\subsection{Because it's all you can do!}
\lecture{15}{01}{2019}

\readings \cite{PsychEgoism} \\ \cite{EthicalEgoism}
\sep

\par{We'll explore two theories which address the issue of wether justice is done for its own sake, or if it comes about just as a by-product of an agent's pursuit of her own self-interest. The first is an ethical theory, namely \ita{ethical egoism (EE)}. EE is normative, i.e. it is concerned with how people ought to behave, and it claims that each person's \ita{ultimate} duty is to pursue his/hers own self-interest*~\mymarginpar{to be read as \ita{the possible course of action in the long run}. Not necessarily hedonic}. It is from this duty that all others duties and obligations are derived from, and if any good to others happens from following this principle, then it is either not purposeful or is calculated (a mere means to one's own end). Note that this necessarily implies that any action which is not in accordance with the pursuit of our own self-interest (if it exists at all) is morally blameworthy. }

\par{ Rachels start by presenting 3 main arguments advanced by the proponents of EE:}

\begin{enumerate}

	\item Looking out for others is self-defeating. 
		\par{Proponents of this type of argument usually claim that since we only have direct access to our own desires, any attempt to act in an altruistic manner is most likely to be unsuccessful; it can even do more damage than good. It is better for all, if we are all left fending for our own interests.}
		\par{Rachels thinks that it seems that the arguments logic is not only flawed, but even if it is granted to be true it does not defend EE. According to the theory, one should never be motivated by other people's interests. But if we are behaving egoistically \ita{because} we are concerned with the general well-being of everyone, then we are not abiding by EE maxim}
		
	\item The denial of the individual
		\par{Libertarians use this principle often, in particular presented \ita{a la} Ayn Rand. Its gist is that by providing charity to someone, one is effect sacrificing something of one's own. Since we have only one life to live, we must regard it to be of the highest importance. Since altruistic actions imply that I regard my life to be less than the person one is helping, then altruistic actions fail to recognize the inherent value of the individual}
		\par{Rachels think that this view is weak, specially because it deals only with extremes. The inherent value of life, of the individual, is to be seen somewhere in between [0-1]. To perform altruistic actions is not to claim \ita{"my life has no value, hence I'll sacrifice it for yours"}; it is instead, to acknowledge that others' interests are also valid, and one must take them into consideration.}
		
	\item Self interest is the fundamental law, from where every other duty is derived
		\par{All those common-sense moral duties (do not cheat, do no harm, etc.) can all be explained be derived from self-interested reasons (e.g. one does not lie, because that erodes trust). The pursuit of self-interest will inevitably lead to the formulation of the golden rule.*~\mymarginpar{*do unto others as you would like to be done onto yourself}}
		\par{Rachels points out that there are cases where it is indeed to our advantage to "do unto others" things which we would not wish be done to us, hence it seems that there are at least some duties that cannot be derived from the "pursuit of self-interest" principle. Furthermore, even if we grant that moral duties can be derived from self-interest, this need not mean that it is \ita{only} because of it that we abey by them, not even that it is the \ita{main} motivating reason.}	
\end{enumerate}

\par{Rachels then presents 3 arguments against EE, one which he deems successful. The first two are, succinctly, (1) that EE cannot account for conflicts of interests. However the egoist may claim that it does indeed, it is only the case that "to resolve a conflict" is not to reach an agreement between the conflicting interests but simply winning it. It is the agent's responsibility to strive to do so. (2) Logically inconsistent, since the same action can be seen as both moral and morally blameworthy (e.g. doing action A leads to a sacrifice of person 1, but a gain to person 2). But as we've seen with Nielsen, the egoist can just claim to be operating from an \ita{agent-relative} viewpoint.To prevent someone from doing their duty (i.e. to fight for their own interests) is not of his concern}

\par{The argument Rachels deems successful is that it seems that EE is \ita{morally arbitrary}. It advocates for people to be discriminated against without any justification. There is no grounds for the distinction "me-rest of the world", the fact that I only have direct access to my own desires, does not mean that I cannot infer from others'* actions, needs, abilities etc. ~\mymarginpar{*philosophical zombies apart} that their interests are comparable to mine.}

\quo{We should care about the interests of other people
for the very same reason we care about our own interests;
for their needs and desires are comparable to our own}

\par{\ita{Psychological Egoism} on the other hand is a descriptive theory. It aims not to to prescribe any behaviour-guiding actions but merely to state/describe how people, as a matter of fact, behave.}

\par{It is widely accepted claim that \ita{one is not morally obliged to do anything that one is not able to do}. Taking this into consideration, it seems that if PE is true, then EE must necessarily be true. Since, if all I \ita{can do} is to pursue my own self-interest, then I really am different in that respect to other people. Hence the the claim that EE is morally arbitrary falls apart}

\par{On its critique of PE, Feinberg argues that the logical status of the theory is unclear. It seems that  PE makes empirical claims, but this is actually a misunderstanding brought upon by our use of natural language. It is not until one properly understands how the terms are being used, than one is able to truly understand what it is that PE is saying, and one does it will become clear that it is paradoxical.}

\par{Feinberg sets himself to find (1) whether FE is True or False ; (2) Is its truth value derived from the fact that its claims are synthetic or analytic? He claims that the problem lies in the way FE uses the term \ita{selfish}, according to Feinberg the term was redefined by the proponents of the theory so as to mean \ita{motivated}. Hence, to call an action selfish becomes the same as saying that it had a purpose, and this purpose was one's own purpose and no one else. This is nothing but a tautology. What FE fails to do is to redefine \ita{unselfish}, and so it incurs the \ita{fallacy of the suppressed correlative}. Therefore what at first seems to be an empirical claim is not, since no evidence can falsify it.}

\newpage
\subsection{Because it's good for you to be good! }
\lecture{16}{01}{2019}

\readings \cite{PlatoRepublicIX} \\ \cite{PlatoRepublic2} \\ \cite{AristotleI}
\sep

\newpage
\subsection{Because of the sort of being that you are! }
\lecture{21}{01}{2019}

\readings \cite{HypotheticalImperatives}
\sep

\newpage
\subsection{Because (duh) it's good! }
\lecture{23}{01}{2019}

\readings \cite{Protagoras} \\ \cite{AristotleVII}
\sep

\newpage
\printbibliography



\end{document}