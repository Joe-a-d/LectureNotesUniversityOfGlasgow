
%% CLASS MANUAL FOUND IN http://blog.poormansmath.net/latex-class-for-lecture-notes/ %%
%% CLASS AUTHOR Stefano Maggiolo %%
\documentclass[english,course]{Notes}

\title{PHILOSOPHY 1B: HOW SHOULD I LIVE?}
\subject{Philosophy}
\author{Joao Almeida-Domingues}
\email{2334590D@student.gla.ac.uk}
\speaker{Jennifer Corns \\ Robert Cowan}
\date{07}{01}{2019}
\dateend{24}{05}{2019}
\place{University of Glasgow}

\usepackage[backend=biber, style=reading]{biblatex}
\bibliography{readings}
\usepackage{csquotes, syllogism}

 %%%%% GENERAL MATHEMATICAL NOTATION SHORTCUTS %%%%%
 
\newcommand{\n}{\mathbb{N}}
\newcommand{\z}{\mathbb{Z}}
\newcommand{\q}{\mathbb{Q}}
\newcommand{\cx}{\mathbb{C}}
\newcommand{\real}{\mathbb{R}}
\newcommand{\field}{\mathbb{F}}
\newcommand{\ita}[1]{\textit{#1}}
\newcommand{\oneton}{\{1,2,3,...,n\}}
\newcommand\ef{\ita{f} }
 %\qedhere to force QED in place
\newcommand\inv[1]{#1^{-1}}
\newcommand\setb[1]{\{#1\}}
\newcommand\en{\ita{n }}
\renewcommand\qedsymbol{QED}
\newcommand\readings{\textbf{Readings:} \\}
\newcommand\sep{\\ \noindent\rule{10cm}{0.8pt} \\}
\newcommand\quo[1]{\begin{displayquote}\ita{\large{#1}}\end{displayquote}}

%%%%%%%%%%%%%%%%PACKAGES%%%%%%%%%%%%%%%%%%%%%%%%%%%%%
\usepackage{lipsum}  
\usepackage{amsmath,amsthm,amssymb,graphicx,mathtools,tikz,pgfplots} %maths
\usepackage{hyperref,framed,color,fancybox} %layout
% framed :  \begin{shaded,frame,snugshade or leftbar} \definecolor{shadecolor}{rgb}{XYZ} to change color
%fancybox: \shadowbox,ovalbox or doublebox
%%%%%%%%%%%%%%%%%%%%%%%%%%%

%%%CLASS SHORTCUTS%%%%
%\lecture{day}{month}{year} for margin note
%\begin{theorem} sdfsdf\end{theorem}
%\begin{proposition} dfsdfs\end{proposition}
%\begin{lemma} dsfsd \end{lemma}
%\begin{corollary} f ffew \end{corollary}
%\begin{definition} fwewef w \end{definition}
%\begin{example} feww e\end{example}
%\begin{exercise} wefwe \end{exercise}
%\begin{remark} wef we \end{remark}
%\begin{fact} wefe \end{fact}
%\begin{problem} wef ew \end{problem}
%\begin{conjecture} ewfew \end{conjecture}
%\begin{claim} few w \end{claim}
%\begin{notation} fewf \end{notation}

\begin{document}
\newpage

\section{Why be good?}

\subsection{Don't be good! Just appear to be}
\lecture{10}{01}{2019}

\readings \cite{Nielsen1984} \\ \cite{PlatoRepublicI}
\sep

\par{ In trying to answer the question "Why be good?", we start by analysing the socratic dialogue in \ita{Plato's - The Republic, Book I}. The main issue raised , which is left unanswered by the end of this text,  can be put forward as \ita{"Must we really be moral, can't we just fake it?"}}

\par{ \mymarginpar{The Republic's analysis}The text opens with Socrates being invited to discuss philosophical matters with a well-to-do elder - Cephalus. For Cephalus, to be good is nothing but to be just which in term means simply , that all one needs to do is to follow simple rules, and is most definitely nothing to worry much about. According to him and his son, one follows the rules, makes some money, pays his debts and that's all there is to it. The rest of the dialogue is pretty much an attempt , on Socrates' part,  to show them that things are not quite as straightforward as they seem to think. To awaken them from their \ita{moral complacency}. Every time Cephalus elaborates in more detail what these rules of his are, Socrates comes up with a counter example which shows that to follow the rule seems to involve some kind of injustice, and that to break it is necessary if good is to be carried out. One particularly good one is when Socrates attacks the necessity of harming one's enemy. Both men agree that to cause harm is to cause the one who is harmed \ita{to deteriorate in excellence}, and since they had agreed at the start that justice is one of the human excellences/virtues, he enquires: how is it that by virtue of doing good, one can make others becomes less good?}

\quo{Can good men by the practice of their virtue make men bad?}

\par{After a while Thrasymacus gets fed up with Socrates' \ita{shtick}, to him Socrates is just another "negative nelly" who goes around town asking questions, destroying people's core beliefs, but offering no alternative. This leads to Socrates (pseudo-)humbly prompting Thrasymacus for his opinion. How does he define justice? Thrasymacus argument is simple. He agrees with the main argument of the first discussion, that people recognise justice by virtue of following certain rules (moral \& legal) . These rules are created by powerful individuals, which are usually taken in high-esteem, or at least are so powerful that their rules remain unchallenged. It only makes sense, Thrasymacus argues, that these rule-makers would make them so that they would be advantageous to them. Again Socrates challenges most definitions used by Thrasymacus and in a similar vain to above he asks: if men are fallible, the laws can be wrong, is it then right to follow them? Thrasymacus agrees that it is not, but he counters by challenging not the action of those who follow them, but the notion of leader itself. A leader who fails, only does so when \ita{"the knwoledge of the craft leaves them"}, and at that moment she is not a leader. He makes a stronger case for himself by appealing to empirical data, or common sense. As if saying; look around,  can't you see that the unjust, the crooks , the ruler of states, regardless of their political believes, wether they be tyrants or democrats, they are the ones with plenty of money and in good health who get to enjoy the pleasures of life. \ita{Book I} ends, with no answer for the issue raised by Thrasymacus, but both Plato and (Plato's) Socrates most definitely do not agree with this view, the whole of \ita{Book I} is an attempt on the part of Plato to show that there are some real questions that one must ask regarding what it means to leave "a good life", it is not by being complacent or a skeptic that one can find out the answers, but they do exist. Exposing them is his goal for the other Books.\mymarginpar{\rem{*Link with the nietzschean view of slave-master morality?}}}

\quo{Those who reproach injustice do so because they are afraid not of doing it but of suffering it*.}

\par{Nielsen agrees that nihilism and ethical scepticism can not be so easily dismissed. He disagrees with the arguments based on the view that to even ask "Why be good?" is non-sensical. His opponents argue that \ita{to say that an act is right just means that one should do it}. He gives enough lead to this argument, he even advances that the ammoral one can very well understand this, from the moral point of view, he claims, \ita{"I should do what everything considered, I regard as right"} can be taken as an analytic truth. But that is not what the immoral agent is asking, he's instead asking "Why should I take the moral point of view at all" , why must moral reasons be taken as overriding reasons for action? Just like Thrasymacus argues above, why must one be compelled by morals and not self-interest? One could argue from an Hobbesian point of view, that moral institutions and the implicit social contract are a necessity if well-being among men is to flourish.\mymarginpar{Link with Nielsen}   But this only shows that \ita{collectively} moral institutions are necessary. But we still hit against the fact discussed above, that those who make these moral institutions can still make them in their own light, with self-interested reasons in mind. Or taking it further, why not just simulate these moral behaviours when in polite company, so as "not to make waves"? Nielsen believes this to be an honest possibility and that it is not unreasonable to leave such a live. In today's age is widely accepted that the less pain and suffering there is, the better, but happiness is not necessarily linked with morality. To be unhappy because one is immoral is contingent and not necessarily true.}

\quo{Will any person, no matter how he is placed in society, or what kind of society he is in, be unhappy or at least less happy, if he is an unprincipled bastard?}

Even if one takes the view that this person would be so awful, that she would be unable to enjoy the pleasures of life that come with the fact that humans are a social animal. This does not mean that this person could not be a sort of \ita{classist amoralist}, as Nielsen puts it. He takes Thrasymacus' arguments further, and shows that if one is powerful enough, one can be a (sophisticaded) tyrant for many (where one treats the dominated classes as mere means to one's ends) while still extending his interest to his peers (where one's able to develop genuine relationships). This is the key distinction between Thrasymacus' all powerful tyrants as \ita{ethical egoists}, and Nielsen's more polished \ita{classist amoralists}. The latter asks "Why not be morally arbitrary?" if it has be shown that "the peace" can still be kept? Nielsen argues that this can not be shown by pure reason, and that the burden of proof lies with those who argue , along Aristotelian lines, that classical immoralism rests in a mistake, in a lack of knowledge.


\newpage
\subsection{To avoid punishment}
\readings \cite{PlatoRepublicII}
\sep
\lecture{14}{01}{2019}

\par{In the start of Book II, Glaucon proposes to Socrates  to strengthen Thrasymachus argument, so that Socrates can argue against it , praising just by itself, and finally show that it is better to be just than unjust. Glaucon first expounds the common view of justice~(\ref{G:what}), he then argues that people who practice it only do so out of necessity, not for its own sake~(\ref{G:selfInterest}). Finally, he concludes that it is not only true that people act unjustly, but they do so because they have strong reasons to do it~(\ref{G:unjustBetter})}

\begin{enumerate}

	\item{What Justice Is?~\label{G:what}}
		\par{According to Glaucon, the common view amongst the people is that to do injustices is naturally good and to suffer them is bad. Hence they \ita{"make laws and covenants, and what the law commands they call lawful and just"}}
		
	\item{To Be Just is a Need only of the Weak~\label{G:selfInterest}}
		\par{It is actually so much worse to suffer them, that the only reason people practice good deeds is simply because they are not powerful enough to enact themselves.To be sufficiently powerful to escape the negative consequences brought upon by the doing of unjust deeds, and not to do them would be seen as stupid by everyone~\mymarginpar{All powerful ring story}}
	 
	\item{The Life of the Unjust is Better~\label{G:unjustBetter}}
		\par{Glaucon then asks Socrates to consider two diametrically opposed characters. One is the most skilled unjust person, who every attempt at injustice remains undetected and is therefore seen as a perfectly just man. The other a sort of martyr; a completely just man who \ita{"doesn't want to be believed to be good but to be so"}. Even though he does no injustice, his fellow men see him as the epitome of it. He again stresses that only a deluded one would prefer the latter's life. That the unjust man's life is better, is not only common sense, but is a matter of fact.
		
		\quo{When fathers speak to their sons, they say that one must be just, as do all the others who have charge of anyone. But they don't praise justice, itself, only the high reputations it leads to and the consequences of \textbf{being thought as just}}}
	
\end{enumerate}

\par{We've addressed this issues in Lecture1 above, it seems that Glaucon is advocating that to appear to be just is enough to avoid punishment and reap the benefits bestowed upon "the moral one". But could there be a reason to \ita{actually be} moral? Adeimantus claims that there are those who argue for this by appealing to \ita{divine punishment}}

\par{Until now we've focused on \ita{social punishment}, and we've raised several arguments that seem to fail to address the distinction between \ita{appearance of being} and \ita{actually being} moral. Given the view of the Gods as omnipresent however, it seems that this appearance/reality gap seems to disappear. In this view, some argue that the Gods give the just the goos they deserve and \ita{"bury the impious and unjust in mud in Hades"}. There are those others however, (in particular the poets) who argue that this doesn't seem to be the case. The Gods seem to like to be entertained, and enjoy human offerings (sacrifices). The unjust man is in a better position to offer such things, hence it seems that as long as he is successful enough, he can buy his eternal redemption. So as long as one believes in divine punishment and this portrayal of the Gods, the unjust is still on the lead}

\par{Socrates defends that justice is a good of the highest kind, good not only for its consequences, but also good in itself. Adeimantus fails to see these, he argues that if justice was like one of those goods (like seeing, hearing, knowing, being healthy...) then one would pursue them for its own sake and we would't need laws and deterrents. Can it be shown that morality is inherent to human rationality, and not just a way of pursuing our own self-interest?~\mymarginpar{Nielsen addressed this concern, see above. Plato won't fully address it until the end of Book IX}}

\newpage


\subsection{Because it's all you can do!}
\lecture{15}{01}{2019}

\readings \cite{PsychEgoism} \\ \cite{EthicalEgoism}
\sep

\par{We'll explore two theories which address the issue of wether justice is done for its own sake, or if it comes about just as a by-product of an agent's pursuit of her own self-interest. The first is an ethical theory, namely \ita{ethical egoism (EE)}. EE is normative, i.e. it is concerned with how people ought to behave, and it claims that each person's \ita{ultimate} duty is to pursue his/hers own self-interest*~\mymarginpar{to be read as \ita{the possible course of action in the long run}. Not necessarily hedonic}. It is from this duty that all others duties and obligations are derived from, and if any good to others happens from following this principle, then it is either not purposeful or is calculated (a mere means to one's own end). Note that this necessarily implies that any action which is not in accordance with the pursuit of our own self-interest (if it exists at all) is morally blameworthy. }

\par{ Rachels start by presenting 3 main arguments advanced by the proponents of EE:}

\begin{enumerate}

	\item Looking out for others is self-defeating. 
		\par{Proponents of this type of argument usually claim that since we only have direct access to our own desires, any attempt to act in an altruistic manner is most likely to be unsuccessful; it can even do more damage than good. It is better for all, if we are all left fending for our own interests.}
		\par{Rachels thinks that it seems that the arguments logic is not only flawed, but even if it is granted to be true it does not defend EE. According to the theory, one should never be motivated by other people's interests. But if we are behaving egoistically \ita{because} we are concerned with the general well-being of everyone, then we are not abiding by EE maxim}
		
	\item The denial of the individual
		\par{Libertarians use this principle often, in particular presented \ita{a la} Ayn Rand. Its gist is that by providing charity to someone, one is effect sacrificing something of one's own. Since we have only one life to live, we must regard it to be of the highest importance. Since altruistic actions imply that I regard my life to be less than the person one is helping, then altruistic actions fail to recognize the inherent value of the individual}
		\par{Rachels think that this view is weak, specially because it deals only with extremes. The inherent value of life, of the individual, is to be seen somewhere in between [0-1]. To perform altruistic actions is not to claim \ita{"my life has no value, hence I'll sacrifice it for yours"}; it is instead, to acknowledge that others' interests are also valid, and one must take them into consideration.}
		
	\item Self interest is the fundamental law, from where every other duty is derived
		\par{All those common-sense moral duties (do not cheat, do no harm, etc.) can all be explained be derived from self-interested reasons (e.g. one does not lie, because that erodes trust). The pursuit of self-interest will inevitably lead to the formulation of the golden rule.*~\mymarginpar{*do unto others as you would like to be done onto yourself}}
		\par{Rachels points out that there are cases where it is indeed to our advantage to "do unto others" things which we would not wish be done to us, hence it seems that there are at least some duties that cannot be derived from the "pursuit of self-interest" principle. Furthermore, even if we grant that moral duties can be derived from self-interest, this need not mean that it is \ita{only} because of it that we abey by them, not even that it is the \ita{main} motivating reason.}	
\end{enumerate}

\par{Rachels then presents 3 arguments against EE, one which he deems successful. The first two are, succinctly, (1) that EE cannot account for conflicts of interests. However the egoist may claim that it does indeed, it is only the case that "to resolve a conflict" is not to reach an agreement between the conflicting interests but simply winning it. It is the agent's responsibility to strive to do so. (2) Logically inconsistent, since the same action can be seen as both moral and morally blameworthy (e.g. doing action A leads to a sacrifice of person 1, but a gain to person 2). But as we've seen with Nielsen, the egoist can just claim to be operating from an \ita{agent-relative} viewpoint.To prevent someone from doing their duty (i.e. to fight for their own interests) is not of his concern}

\par{The argument Rachels deems successful is that it seems that EE is \ita{morally arbitrary}. It advocates for people to be discriminated against without any justification. There is no grounds for the distinction "me-rest of the world", the fact that I only have direct access to my own desires, does not mean that I cannot infer from others'* actions, needs, abilities etc. ~\mymarginpar{*philosophical zombies apart} that their interests are comparable to mine.}

\quo{We should care about the interests of other people
for the very same reason we care about our own interests;
for their needs and desires are comparable to our own}

\par{\ita{Psychological Egoism} on the other hand is a descriptive theory. It aims not to to prescribe any behaviour-guiding actions but merely to state/describe how people, as a matter of fact, behave.}

\par{It is widely accepted claim that \ita{one is not morally obliged to do anything that one is not able to do}. Taking this into consideration, it seems that if PE is true, then EE must necessarily be true. Since, if all I \ita{can do} is to pursue my own self-interest, then I really am different in that respect to other people. Hence the the claim that EE is morally arbitrary falls apart}

\par{On its critique of PE, Feinberg argues that the logical status of the theory is unclear. It seems that  PE makes empirical claims, but this is actually a misunderstanding brought upon by our use of natural language. It is not until one properly understands how the terms are being used, than one is able to truly understand what it is that PE is saying, and one does it will become clear that it is paradoxical.}

\par{Feinberg sets himself to find (1) whether FE is True or False ; (2) Is its truth value derived from the fact that its claims are synthetic or analytic? He claims that the problem lies in the way FE uses the term \ita{selfish}, according to Feinberg the term was redefined by the proponents of the theory so as to mean \ita{motivated}. Hence, to call an action selfish becomes the same as saying that it had a purpose, and this purpose was one's own purpose and no one else's. This is nothing but a tautology. What FE fails to do is to redefine \ita{unselfish}, and so it incurs the \ita{fallacy of the suppressed correlative}. Therefore what at first seems to be an empirical claim is not, since no evidence can falsify it.}

\newpage
\subsection{Because it's (prudentially) good for you to be (morally) good! }
\lecture{16}{01}{2019}

\readings \cite{PlatoRepublicIX} \\ \cite{PlatoRepublic2} \\ \cite{AristotleI}
\sep

\par{Common sense morality sees living a good life as doing good deeds, for socrates this is wrong. To live a good and just life is to be a certain kind of person $\implies$ external Vs internal characteristics}

\rem{Plato uses the greek word \ita{psuche}, which translates roughly to something like \ita{life} or \ita{being}. In the text this is often translated as \ita{soul}, but is not to be read as in the cartesian sense of \ita{soul}. Instead 'soul' and 'person' will be used interchangeably}

\par{Following the same rationale used in the construction of his \ita{kallipolis} - the theoretical perfectly good city -( \ita{Book II} ), Socrates in the interviewing books compares and contrasts the five "city-character types" pairs~\mymarginpar{king, timocrat, oligarch, democrat, tyrant}. In \ita{Book IX} he starts by providing us with a long description of what a tyrannical man in fact represents. In discourse with Glaucon and Adeimantus, he advances that the tyrant is the most unjust and wretched of all men, for even though all men have desires/feel the force of the passions, the tyrant is made \ita{slave} by them. A tyrant's soul must then be \ita{"poor and unsatisfiable"}. Within the tyrants, he presents us with a further level of distinction, those who live a private life (who are never able to grab power to rule over others) and those who don't. The latter is by far the most wretched, for just like a slave-owner would, outside the protection of his city , be at the mercy of his angry slaves, and would be made to free them out of fear. So must an actual tyrant be torn part, and ultimately guided by his passions. Yet, even though he can't govern himself, he makes futile attempts to rule over others which frustrates him even further.}

\quo{And, if indeed his state is like that of the city he rules, then he's full of fear, convulsions, and pains throughout his life [\dots] he is inevitably envious, untrustworthy, unjust, friendless, impious host and nurse to every kind of vice, and his ruling makes him even more so. And because of all these, he is extremely unfortunate and goes on to make those near him like himself}

\par{Following this, they conclude that the \ita{philosopher-king}*, the enlightened man whose character most resembles the \ita{kallipolis}~\mymarginpar{*Whose description is developed in detail between II-VIII must ,by necessity of being diametrically opposed to the tyrant, be the most virtuous and happy}}

\quo{[\dots] the most kingly, who rules like a king over himself}

\newpage
\par{Having honed on the meaning of tyrant, and ranked the five character types in order of virtue and happiness Socrates starts to develop his second argument by advancing the theory that the soul is divide into 3 key and distinct parts. They are so organised, because each provide a \ita{distinct} set of motivations that can explain people's actions.}

\begin{enumerate}
	\item ~\mymarginpar{Tripartition of the soul}Appetitive \ita{(money and profit-loving)}: can be seen as the self-indulgent part of the soul. According to plato it is the largest, strongest and most violent part when seeking particular objects of desire (e.g. money, sex, food , \dots)
	\item Spirited \ita{(victory and honor-loving)}: the emotional and "life-goal setting" part, which is \ita{"wholly dedicated to the pursuit of control, victory, and high repute"}. It is responsible for providing the necessary motivation so that long-term accomplishments can be achieved.
	\item Rational \ita{(learning-loving, philosophical)}: includes using rational activities in search of truth and wisdom/enlightenment. The motivation from the spirited part is essential for these activities, but ultimately, argues Plato, the rational part \ita{should} be the major \ita{ruling force} for a person as a whole
\end{enumerate}

\par{From this split it follows, that people who are more inclined to "feed" one of them more, will find pleasure in different forms, and when asked to judge how pleasant their life is will deem theirs to be the most happy/pleasurable. Although this is naturally to be the case, it is not how one should go about trying to found out which one \ita{truly} is the most pleasurable. Socrates argues, that if we are to judge correctly we must use experience, reason and argument. And not only is the Philosopher the most well-versed in this matters, he also has the unique epistemically advantage of having experienced all other pleasures~\mymarginpar{Cave Allegory; Link with Mill's Higher/Lower pleasures} while growing up while the others need not necessarily ever have experienced the joys of learning}

\quo{Then, as far as experience goes, he is the finest judge of the three. [\dots] Moreover, the instrument one must use to judge isn't the instrument of a profit-lover or an honour-lover but a philosopher. [\dots] The praise of a wisdom-lover and argument-lover is necessarily [the] truest}

\par{Socrates then ends \ita{Book IX} by first hammering down this notion of ignorance as a fault ("empty state") of the soul. Those who take joy in "earthly pleasures", who are enslaved by their passions or pride, only deem these as pleasurable for the only other they know is pain and when compared to it they seem immensely pleasurable indeed. They are however not aware that there's a higher level yet~\mymarginpar{story about the man who only knows down and middle} . The level of pleasures which feels the soul with those things which are \ita{always the same, immortal, and true} ~\mymarginpar{this notion of purer pleasures is heavily dependent on the metaphysical framework developed in V-VII}. By illustrating the soul as a mythical beast \ita{"where many kinds of things are said to have grown together into one"}composed by a multi-headed beast, a lion and a human; He argues that it would be unreasonable to expect the soul to thrive, or indeed to survive, if one first feeds the multiform beast, then the lion, while starving the human being~\mymarginpar{Conflict within the soul of the unjust}. One should instead do the opposite, and in so doing, one will be able to accustom them to each other and make them friendly.}

\par{Having established the lack of virtue as the \ita{shameful} things as those which \ita{"enslave the human parts of our nature to the savage"} he responds directly to the Thrasymachus' claim, that it is better to be unjust lest one find's oneself in the end of an unjust one, by showing that the the enlightened ruler, the one who is virtuous, rules not because he wants to power over others, or out of fear, but out of duty. Because it is better for everyone.}

\quo{To ensure that someone like that [who cannot control the beasts within him] is ruled by something similar to what rules the best person, we say that he ought to be the slave of that best person who has a divine ruler within himself. It isn't to harm the slave that we say he must be ruled, [\dots] but because it is better for everyone to be ruled by divine reason, \textbf{preferably within himself} and his own, otherwise imposed from without, so that as far as possible all will be alike and friends, governed by the same thing}

\par{So it seems as if Plato is claiming that even the most skilled of the malefactors, who appears just to others and who is wealthy and indulges in all sorts of bodily pleasure, will be so conflicted internally that will have the most miserable of existences. So it is \ita{prudentially} good to be morally good, even though to achieve the moral good is our final end (an end in itself)~\mymarginpar{link with Aristotle's eudaimonia}}
\\
\par{This response fails to address the objection we've already seen above raised by Nielsen, that to be unhappy because one is immoral is contingent and not necessarily true. Is there empirical proof that a mentally strong immoralist would be plagued by such guilt and internal conflict that he wouldn't enjoy his life?}
%Prudential should: this kind of ought/should depends upon your having a certain self-interested goal or aim
 
\newpage
\subsection{Because of the sort of being that you are! }
\lecture{21}{01}{2019}

\readings \cite{HypotheticalImperatives}
\sep

\quo{Even if it's not prudentially good to be morally good, is there something about the sort of creatures we are that gives us reasons to be moral? \\}

\par{Plato's move against the single-minded pursuit of self-interest was that to do so would be detrimental to one's soul, and would make one slave to one's desires and passions. Instead of pursuing prudential goods as ends in themselves, one should instead strive to maximise one's virtues and the former would come as a by-product of doing so.However, by arguing from human nature (in particular that it is by doing $x$ that we get our desires satisfied) Plato wants us to derive a \ita{normative} claim from a \ita{descriptive} one. Hence, empirical proof is needed to justify it.}

\par{One first objection that can be raised is that, acting immorally is not \ita{always} contrary to one's flourishing. A stronger one would be that maybe from Plato's point of view no human being could live in good conscience, and flourish by not cultivating his/her virtues, but is his point of view true? So that  by combining the 2, maybe only the lesser claim \ita{"It is only prudentially good to be moral sometimes, for some people"} holds. Unless some universal reason exists which holds for \ita{everyone} at \ita{all times}, then there it is not unreasonable to consider the aforementioned claim true. }

\par{One of the main proponents of the existence of such a reason was Immanuel Kant, for him moral actions are not those which can derived from self-interest, or with regards to their possible consequences, but are instead which are done \ita{out of duty} (without self-interest or inclination). To determine if an action is done from duty one cannot then look outside into the world, i.e. moral actions \ita{are not contingent} on anything else. So for Kant an action done from duty are those which following universalisable maxims, what he called \ita{categorical imperative}:}
\begin{enumerate}
	\item Standard of Rationality;
	\item Objective, Rationally Necessary, Unconditional
	\item Must Always be obeyed
	\item Must be of an agent?s own authorship
\end{enumerate}

\par{If an universalisation of a maxim fails to be universalised due to being logically impossible to do so like for example the maxim \ita{"everyone should own slaves"}, which leads to a \ita{contradiction in conception}, i.e. it is inconceivable to think of a world where it could be true, then \ita{perfect duties} can be derived from it - \ita{"I should not own slaves"} ; \ita{"Owning slaves is wrong}. The categorical imperative is then the universal reason, according to Kant, that we were searching to justify Plato's claims, and the perfect duties the actions which were derived from such undeniable axioms via reason*\mymarginpar{*can be compared as the process by which an educated philosopher king would come up with his "truths"} , those that \ita{should not} be broken \\ }
	
\par{Philippa Foot however, thinks that Kant's , or Plato's philosopher kings for that matter, give too much and an unstained force to the moral \ita{should}. Foot argues that Kant's necessary distinction between hypothetical and categorical imperatives is unfounded, and modern philosophers often follow in his footsteps by using such terms as \ita{unconditional requirement} when discussing what Foot calls the "feeling" we get that certain rules cannot be broken. However, she counters that there are other uses of "should" being used non-hypothetically, where the same force we apply to moral imperatives does not seem to apply (e.g. rules of etiquette) , i.e. we do not classify them as categorical even though they are not done on the grounds of self-interest. Furthermore, much like Nielsen, Foot advances that those who do not follow moral rules because they see no reason to obey them may be \ita{"convicted of villainy but not of inconsistency"}. Their actions are not irrational, for they are not self-defeating.}

\quo{There is no difficulty about the idea that we fell we have to behave morally, and given the psychological conditions of the learning of moral behaviour it is natural that we should have such feelings. What we cannot do is quote them in support of the doctrine of the categorical imperative}

\par{Foot's refusal of the 'binding force of morality' puts forward morality as a \ita{system of hypothetical imperatives}. That we \ita{feel} we must do good (and often do) is a consequence of our education, and cannot be used in support of universal maxims. However, this is not an attempt to abolish morally behaviour. The control that we feel the categorical imperative gives us, should instead be replaced by the empirical observation that people often band together to fight hard for moral ends \mymarginpar{How is this not circular reasoning? People fight hard because they were educated in x way, but their education was based on the common place view of morality as a force majeure. Hence, if special status of morality does not hold, why should people not question their education?}

\quo{If we put this theory of human action aside, and allow as ends the things that seem to be the ends, the picture changes. It will surely be allowed that quite apart from thoughts of duty a man may care about the suffering of others, having a sense of identification with them, and wanting to help if he can. Of course he must want not the reputation of charity, nor even a gratifying role helping others, but, quite simply, their good.}

\newpage
\subsection{Because (duh) it's good! }
\lecture{24}{01}{2019}

\readings \cite{Protagoras} \\ \cite{AristotleVII}
\sep

\par{Foot defended that  we cannot ascribe a special status to moral rules, hence we cannot call upon it as a reason for someone to obey them. Someone with ends other than moral ones can not be called irrational, for they are doing nothing disadvantageous to themselves.}
\par{Plato however, disagrees with Foot's view. As we've touched already when analysing his response to Thrasymachus, Plato sees ignorance as an \ita{empty state of the soul}, and rationality as the way for true enlightenment. Hence, for him one only does bad deeds out of ignorance.}
\par{In \ita{Protagoras} Plato argues against the view that people only make wrong choices when they are weak-willed, (what Aristotle called \ita{akrasia}) and their passions take over their rational will. His argument is that regardless of what people might call it - pleasure, joy and pains, suffering - in the end all actions pursued by a rational agent like humans can be reduced to the pursuit of good or evil. The process of choosing which action to pursue is a matter for the intellect, in particular the part which occupies itself with comparing and contrasting the level of pleasure and pain two different actions will bring \ita{in the long run}.

\quo{What is your name for what we called being worsted by pleasure? [\dots] We take it that you say this happens to you when, for example, you are overcome by the desire of food or drink or sex - which are pleasant things - and though you recognize them as evil, nevertheless indulge in them. [\dots] In what respect do you call them evil? [\dots] [are] they not evil on account of the actual momentary pleasures which they produce, but on account of their consequences, disease and the rest?}

\par{He advances a simple heuristic, when weighing 2 pleasures one always choose the one which will bring "the greater and more" and the opposite is true when comparing 2 pains. When measuring a pleasure against a pain however, one always chooses pleasure over pain. Having established this, he then argues that it is non-sensical to claim that one \ita{knowingly} chooses to do evil because he is overcome by good. This could only happen, he advances, if the process of weighing the actions has somehow gone astray. Most often it happens that, just like one judges things which are near to be greater than those which are far away, so it happens that one judges things which are more pleasurable in the future to be less so when compared to immediate pleasures. This is a fault of the intellect, one cannot therefore say that one \ita{knew} that it was choosing to do evil, instead one simply \ita{believed} it to be the case, and so chose to do it out of ignorance}
\newpage
\quo{Well then, since our salvation in life has turned out to lie in the correct choice of pleasure and pain - more or less, greater or smaller, nearer or more distant - is it not in the first place a question of measurement, consisting as it does in a consideration of relative excess, defect or equality? [\dots] And if so, it must be a special skill or branch of knowledge}

\newpage

\section{What is good?}
\lecture{30}{01}{2019}

\readings \cite[Chapter 2]{driver_2014} 
\sep

\par{In the last section we were looking at the question of "Why be good?", i.e. what reasons might one have to be good. In this section we'll be looking at what does "to be good" mean, what are "the right things" and is it about them that make it right?}
\par{ When we judge wether an action is right or wrong we perform a \ita{moral evaluation}. Moral evaluations seem to be verifiable and objective, we believe that by reasoning and arguing we can reach some sort of moral truth. Note however that, when we undergo this process we often consider our intuitions}

\subsection{Divine Command Theory (DCT)}

\par{DCT is a meta-ethics theory which defends that morality is dependent on God. In particular, God's commands are \ita{necessary and sufficient} for something to be deemed good or bad, and they also \ita{make} them have moral features, i.e. they are right or wrong precisely \ita{because} God commanded so}

\par{Two common \textbf{misconceptions} (which often lead to straw man arguments against it) are that (1) Atheists cannot perform morally right actions; (2) Humans would have no concept of morality without knowledge of God's commands. However, given the view of God as a creator, what the theory claims is that we were endowed by God with certain features (e.g. rationality, ability to feel pain etc.) to recognize actions as good; It is in our nature. Hence an atheist can simultaneous not believe in god and be charitable, for example \\}

\par{What motivations do we have to believe the theory to be true?}
\begin{enumerate}
	\item{From a theistic p.o.v it makes sense to believe that the creator of the universe and its laws, would also be the creator of our moral rules}
	\item{It accounts for the intuitively objective nature of morality, since it puts the onus of defining it on some superlative entity}
	\item{It accounts for the common sense view, discussed in the previous section, that moral rules are \ita{overriding} or categorical. Since, the existence of a powerful lawgiver compels us to follow them}
\end{enumerate}

\subsubsection{The \ita{Eutyphro} Dilema}

\quo{Is the pious loved by the gods because it is pious, or is it pious because it is loved by the gods? ---Plato}

\par{DCT proponents must subscribe to the latter view, since the former would mean abdicating the view that morality is dependent on God. However, by subscribing to the latter, they encounter an obvious objection - \ita{the "anything goes" objection} essentially raises the point that by claiming that whatever God commands is right, then it is conceivable that certain actions we all see as \ita{just} wrong (e.g. torturing a baby), could be made to be right \ita{if} God commanded them.}

\syllog{(P1) If things are right and wrong because of God's commands, then 
gratuitous torture could have been right (had God commanded it).}
{(P2) Gratuitous torture could never be right.}
{(C) It is not the case that things are right and wrong because of God' 
commands.}

\par{The common way to tackle the argument, is by claiming that there is something about the nature of God as a perfectly good, omnipresent, omnipotent and omniscient being which is incompatible with the conceivability of moral calamities. However, this leads to objections against either his omnipotence or goodness, usually because by saying that God "can but won't" implies that God's actions are not what \ita{makes} something right or wrong, instead, She looks at something external and simply refrains from commanding it}


\lecture{31}{01}{2019}


\newpage
\subsection{Moral Relativism}
\readings \cite[Chapter 1]{driver_2014} 
\sep

\par{Moral Relativism comes in many flavours, when considered from the pov of individuals the theory gathers very few supporters, a stronger formulation is from the perspective of cultures/societies. In short, MR claims that what is necessary and sufficient for things to be morally right , and what makes them right are \ita{relative} to each society and established by agreements between their members}

\par{The main motivation for relativists is that it \ita{seems} to reflect actual moral practice, i.e. the descriptive claim that there exist cultural and moral disagreements between cultures is not only verifiable but indeed true, and this is proof that no universal moral truths exist. Furthermore, in the light of the social rights movement, this view is particularly attractive for some because it justifies tolerance of different moral views.}
\par{This simple formulation of the theory however seems to face one of the objections seen above, the fact that things which most (if not all) of us see as plainly wrong, across cultures, could be made to be right as long as enough people agree that it is. Most people agree that this could not be, there are indeed actions which are just plainly wrong, and to tolerate their practice should not be seen as virtuous.} 
\par{Furthermore, Rachels argues that the fact that there is moral disagreement, does not mean that Universalism is incorrect. Some cultures, or even all cultures, might be plainly mistaken about what right is. In addition, when philosophers talk of the existence of universal moral claims, they talk of prescriptive claims, which cannot be refuted by simply looking at how things are. A stronger point yet, is that most cultures do not in fact differ that much when it comes to their fundamental moral principles (e.g. killing, stealing, breaking promises is wrong etc), it could simply be the case that different circumstances lead to different moral \ita{practices}}




\newpage
\subsection{Utilitarianism and the Value of Happiness}
\readings \cite[Chapter 3]{driver_2014} \\ \cite[Chapter 12]{darwall_1998} \\ \cite{mccord}
\sep

\lecture{04}{02}{2019}

\par{Utilitarians claim that when evaluating the moral value of an action, one should look at its consequences. If an action promotes general happiness than that action is said to be morally right. \ita{Jeremy Bentham} is said to be one of the first proponents of this theory, in particular of \ita{Hedonic Act} Utilitarianism, since he equated ac


\newpage
\subsection{Utilitarianism and the Right Action}
\readings \cite[Chapter 4]{driver_2014} \\ \cite[Chapter 13]{darwall_1998}
\sep

\lecture{05}{02}{2019}


\newpage
\subsection{Kantian Ethics and the Values of Freedom}
\readings \cite[Chapter 5]{driver_2014} \\ \cite{mccord} \\ \cite[Chapter 14]{darwall_1998} \\ \cite["Laws of Freedom]{guyer}
\sep

\newpage
\subsection{Kantian Ethics and the Right Action}
\readings \cite[Chapter 5]{driver_2014} \\ \cite["Laws of Freedom"]{guyer}
 \\ \cite[Section II]{kant_groundworks} \\ \cite[Chapter 15]{darwall_1998}
 \sep
 
 



\newpage
\printbibliography



\end{document}