
%% CLASS MANUAL FOUND IN http://blog.poormansmath.net/latex-class-for-lecture-notes/ %%
%% CLASS AUTHOR Stefano Maggiolo %%

%%%%%%%%%%%%%%%  TITLE PAGE  %%%%%%%%%%%%%%%%%%%
\documentclass[english,course]{Notes}
\title{COURSE CODE}
\subject{SUBJECT AREA}
\author{Joao Almeida-Domingues}
\email{2334590D@student.gla.ac.uk}
\speaker{LECTURER}
\date{00}{00}{1990}
\dateend{07}{07}{1994}
\place{University of Glasgow}


%%%%%%%%%%%% BIB CONFIG %%%%%%%%%%%%%%%
\usepackage[backend=biber, style=reading]{biblatex} 

\bibliography{} %add bib file name

%%%%%%%%%%% LAYOUT  %%%%%%%%%%%%%%%

\renewcommand{\abstractname}{\vspace{3\baselineskip}} %hack to remove abstract


%%%%%%%%%%%%%%%%%%%%%%%%%%%%%%%%

%%%%%%%%%%%%%% KEEP HERE (conflict when in class) %%%%%%%%%%%%%%%%%%%%

 %%%%%MATRICES
    
    \let\mat=\spalignmat
    \let\amat=\spalignaugmat
    \let\vec=\spalignvector
    
%%%%%% row ops
    \newcommand\ro[2]{\xrightarrow[#2]{#1}}
%%%%%%%%%%%%%%%%%%%%%%%%%%%%%%%%%%%%%%%%%%%%%%%%%%%%%

%%%%%%%%%%%%%  PACKAGES (NOT INCLUDED IN CLASS) %%%%%%%%%%%%%%
\usepackage[delims={[]}]{spalign}

%%%%%%%%%%%%%%%% ALGORITHM TEMPLATE %%%%%%%%%%%%%%%%%%%%

%\begin{algorithm}[H]
%\SetAlgoLined\KwData{this text}
%\KwResult{how to write algorithm with \LaTeX2e }initialization\;
%\While{not at end of this document}
%	{read current\;\eIf{understand}{go to next section\;current section becomes this one\;}{go back to the beginning of current section\;}}
%	\caption{How to write algorithms}
%\end{algorithm}

%%%%%%%%%%%%%%%%%%%%%%%%%%%%%%%%%%%%%%%%%%%%%%%%%%%%%

\begin{document}

%%%%%%%%%%%%%%  DISCLAIMER  %%%%%%%%%%%%%%%%%%%%%

\begin{abstract}
	\par{These lecture notes were collated by me from a mixture of sources , the two main sources being the lecture notes provided by the lecturer and the 
content presented in-lecture. All other referenced material (if used) can be found in the \ita{Bibliography} and \ita{References} sections.}
	\par{The primary goal of these notes is to function as a succinct but comprehensive revision aid, hence if you came by them via a search engine , please note 
that they're not intended to be a reflection of the quality of the materials referenced or the content lectured.}
	\par{Lastly, with regards to formatting, the pdf doc was typeset in \LaTeX , using a modified version of Stefano Maggiolo's \href{http://blog.poormansmath.net/
latex-class-for-lecture-notes/}{\underline{\textcolor{blue}{class}}}}
\end{abstract}
\newpage

%%%%%%%%%%%%% LECTURES %%%%%%%%%%%%%%%%%%%%%%%




%%%%%%%%%%%%%%  BIBLIOGRAPHY  %%%%%%%%%%%%%%%%%%%
\newpage
\nocite{*}
\printbibliography

%%%%%%%%%%%%%%%%%%%%%%%%%%%%%%%%%%%%%%%%%%

\end{document}
