
\section{Modelling}

\lecture{13}{01}{20}
	\key{Object}{Encapsulation}{Polymorphism}{Inheritance}{Abstraction}{Class}{Methods}{Attributes}{End-User}

	\par{This course will focus on the basic concepts of OOP. The main idea
being that one can model real world objects as abstractions in software. We take
the object's attributes and functions and convert them into a single entity
consisting of data and methods which operate on that data}

\subsection{3 Principles of OOP}

	\defn{Encapsulation}{when data and operations exist within the same entity}
	\defn{Inheritance}{classes can inherit attributes and methods from other
classes}
	\defn{Polymorphism}{ability of an entity to take many forms}

	\par{As seen in JP2, the data and methods are defined within the class, and
are often protected from outside access unless via getter and setters. If a
class is a subset of another class then it inherits all (or most) of its
behaviours and attributes and so it can be \ita{subclassed}. This ability of the
superclass to take many forms depending on which child is called is one of the
crucial aspects of OOP. When methods are \ita{overridden} by child class a
method of an object reference to a superclass is invoked at compile time, and it
is later dispatched to the overridden of the specific class instance at run time} 

\subsection{OO Design}

	\par{Software design relies on a symbiotic relation between the end-user and
the designer. Often one is given a spec/problem statement by a client, or given
a certain user story (recall HCI:1F) and from there the designer will use its
modelling knowledge to interpret it in the light of OOP paradigms}
	
\begin{enumerate}
	\item Identify real world objects (look at the nouns in the spec)
	\item Identify relationships between objects
		\begin{enumerate}
			\item[] Generalization : Abstract common features (e.g \ita{move})
			\item[] Containment : Object A $\subseteq$ B (e.g \ita{Dog} $\subset$
					\ita{Animal})
			\item[] Multiplicity : Quantity relation (e.g Dog (1)
					$\leftrightarrow$ (Many) Paws)
		\end{enumerate}
	\item{Identify operations and associate them with objects (this is usually
		done by looking at the verbs in the spec)}
	\item{Create an Interface}
		\begin{itemize}
			\item[]{it is essentially a contract which guaranteed that each object
					represented by a given class will behave in a specified manner}
			\item[]{it must include a \ita{return type} ,
\ita{purpose/description} , \ita{pre-conditions} , i.e what must be true prior
to the method being called , \ita{post-conditions} , what must be true when
returning}
		\end{itemize}
		\item{Object Encapsulation , which describes how objects communicate
		via operations and how this affects the end-user} 
\end{enumerate}


