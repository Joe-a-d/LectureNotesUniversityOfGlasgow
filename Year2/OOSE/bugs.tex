\section{Software Quality}

	\key{Static Analysis}{Bug
Patterns}{Soundness}{Precision}{Debugger}{Bug}{Debugging
Techniques}{Bytecode}{Source Code}{Bug Triage}{Software Reliability}

	\subsection{Introduction}

		\defn{Software Reliability}{denotes a product's trustworthiness or
dependability}

	\rem{software reliability varies \ita{directly} with the number of bugs}

		\defn{Bug}{a failure in a computer system that produces an incorrect or
unexpected result}

	\par{We can define a bug in terms of which outcome it produces given its
spec. In particular, we classify something as a bug, if one of the following
occurs}

		\begin{enumerate}\mymarginpar{$S$ is spec , $O$ is outcome, and $X$ is a
behaviour}
			\item $X \in S \land X \not\in O$
			\item $\neg X \in S \land X \in O$ 
			\item $X \not\in S \land X \in O$
			\item broken spec and broken outcome ; i.e. doesn't do X and it
isn't mentioned \ita{but it should}
			\item clunky software, broken from an untrained user's perspective 
		\end{enumerate}
	
		\subsubsection{Bug Causes}

		\par{The major cause of bugs occur at the specification level, followed
by the design stage and actual code}
		\par{At the spec level is often due to a breakdown in communication, or
a lack of formality in its implementation (e.g. not written, incomplete, sketch
only). The design stage is often overlooked or rushed, leading to
inappropriate modelling, or a lack of modeling tools. Coding errors can occur
due to a complex nature of the software, or simpler things like poor
documentation or limited time for project completion leading to rushed work}

	\rem{some coding codes examples are wrong messages, overflow errors,
override errors, misuse local variable, syntax ...}

	\rem{the cost of fixing a bug grows linearly with the life of a project,
from spec design to release}

	\subsection{Debugging Techniques}

	\defn{Bug Triaging}{is the assignment of a bug to the most
appropriate/capable developer who will fix it}

	\par{The triager needs to be aware of both the project, the bug reports and
the team involved in the project. A general workflow for bug is as follows:}

		\begin{enumerate}
			\item a bug is reported
			\item bug is assigned to manager for triage
			\item if fixed END , else assign to developer to fix
			\item re-test new code ; if fixed END else goto 2
		\end{enumerate}
 
	\defn{Static Analysis}{a method of a computer program debugging that is done
by examining the code without executing the program. It provides an
understanding of the code structure}

	\defn{Threat Modelling}{inspect at design stage, and draw out possible error
paths}
	
	\defn{Manual Code Reviews}{read source code, and reason based on your
knowledge whether bugs can occur}

	\rem{Note that this process can be quite subjective, depending on the
inspector}

	\defn{Automated Tools}{tools which parse program code, and automatically
detect common bugs}

	\rem{AT is particularly efficient for finding common bugs/bug patterns}

	\defn{Bug Pattern}{recurring correlations between signalled errors and
underlying bugs}

	\begin{enumerate}
		\item\textbf{Predictable random number generator} \par{given the
\ita{pseudo} nature of random generators, it is easy for a malicious user to get
hold of the next random number in a series. In Java, there's a better way to
generate random numbers by using the \texttt{SecureRandom} library instead of
the \texttt{Random}}
		\item\textbf{Object Deserialization} \par{object deserialization of
untrusted data can lead to remote code execution (e.g open attachments in
untrusted email sources).}
		\par{It is best to avoid untrusted sources, but when not possible, then
input validation is robustly applied for sanitation. Thus mitigating any
malicious input strings being converted to executable binary}

		\item\textbf{Trust Boundary Violation}\par{when the trust boundary is
blurred, and data is passed over the boundary without proper validation}
		\item\textbf{Null Pointer Exception}\par{when an object which does not
exist is accessed}
		\item\textbf{Infinite Recursion}a call to a recursive function that
never reaches the base case
	\end{enumerate}	

	\defn{Deserialization}{taking structured data (e.g JSON) and rebuild it into
an object} 

	\defn{Trust Boundary}{an imaginary line drawn through logical pieces of a
program where on one side we assume that the data is trustworthy, and on the other
untrustworthy}	

%%% COPY BUG CATEGORIES %%%

\par{Manually examine source code can be labour intensive and subjective
depending on the inspector}

	\defn{Debugger}{special program used to analyse other programs in order to
find bugs}

	\par{The debugger will analyse program code (e.g source, bytecode , etc) and
will analyse it in terms of the correctness of its statements, control flow,
method class , etc. without actually executing the code}
	
	\subsubsection{The Limitis of Static Analysis}
		
		\par{We can't really tell whether a program $P$ has some property $Q$ ,
instead we \ita{approximate} $Q$ in analysis $P$}	

	\rem{Recall \ita{Church-Turing's} introduction of the \ita{Halting
Problem}~\cite{turing1937}~\cite{church1936}}

	\defn{Soundness}{ if sound then an alert is
raised; unsound means that the detection system can generate false negatives
}

	\defn{Precision}{measures accuracy of bug detection. If precise then every
bug alert is an actual bug ; imprecise implies possible false positives}

	\par{Most analysis tools involve a trade-off between soundness, precision
and execution time. Though, the majority of tools are \ita{conservative} in
nature, i.e soundness is often preferred}

	\par{So, depending on the project, one can approximate towards:}
		\begin{itemize}
			\item\textbf{Completeness}\par{where the detection tool is
designed in a such a way so as to \ita{overestimate} possible program behaviours ;
with the drawback that the false positives might overshadow the real bugs}
			\item\textbf{Soundness}\par{where the detection tool
\ita{underestimates} the possible program behaviours ; with the drawback that the analysis is
incomplete}
		\end{itemize}

	\rem{in practice a \ita{balanced approximation}, which is neither sound nor
complete, allows for enough flexibility so that the program behaviour is more
easily estimated}
 
