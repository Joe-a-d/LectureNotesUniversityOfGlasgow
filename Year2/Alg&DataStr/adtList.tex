\section{List}

		\par{The ADTs presented above could all be implemented using either an array or a linked list, due to fact that they represent a \ita{linearly ordered} sequence of elements. The \texttt{List} however provides more general support for manipulating a sequence of countable elements at arbitrary positions, hence an efficient implementation with either an array or linked list is challenging}

		\rem{fundamental data type in most functional languages}

		\subsection{operations}
				\begin{enumerate}
						\item[] GET(L,i)
						\item[] SET(L,i,x) : add x at i, return replaced 
						\item[] ADD(L,x)
						\item[] ADD-AT(L,i,x) : add x at i, shift right
						\item[] REMOVE(L,i) : remove and shift left
						\item[] SIZE()
						\item[] isEmpty()
						\item[] APPEND(L1,L2)
				\end{enumerate}


		\subsection{Analysis}
		
			\begin{table}[H]
				\begin{tabular}{lcccc}
					\cline{2-5}
					\multicolumn{1}{l|}{}                        & \multicolumn{1}{c|}{\textbf{GET/SET (Indexing)}} & \multicolumn{1}{c|}{\textbf{ADD}} & \multicolumn{1}{c|}{\textbf{ADD-AT/REMOVE}} & \multicolumn{1}{c|}{\textbf{APPEND}}  \\ \hline
					\multicolumn{1}{|l|}{\textbf{Dynamic Array}} & \multicolumn{1}{c|}{$O(1)$} & \multicolumn{1}{c|}{$O(n)$}       & \multicolumn{1}{c|}{$O(n)$}                 & \multicolumn{1}{c|}{\$O(n\_1 + n\_2)} \\ \hline
					& \multicolumn{1}{c|}{$O(1)$}       & \multicolumn{1}{c|}{$O(n)$} & \multicolumn{1}{c|}{$O(1)$}           \\ \hline & \multicolumn{1}{l}{}& \multicolumn{1}{l}{}              & \multicolumn{1}{l}{}                        & \multicolumn{1}{l}{}                 
				\end{tabular}
			\end{table}
