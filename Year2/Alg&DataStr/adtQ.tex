\subsection{Queue}

		\defn{Queue}{ is a collection of objects which are inserted and deleted following the \ita{FIFO} principle}

		\rem{sharing services (e.g printer) , waiting lists, multitasking}

		\subsubsection{Operations}
				\begin{itemize}
						\item Essential Update Methods
						\begin{enumerate}
								\item enqueue
								\item dequeue
						\end{enumerate}
				\end{itemize}
				\begin{itemize}
						\item Accessor Methods
						\begin{enumerate}
								\item first
								\item size
								\item isEmpty
						\end{enumerate}
				\end{itemize}
				
		\subsubsection{Array Implementation}
				
				\par{Using a \ita{wrap-around} array, let \texttt{Q.head , Q.tail} represent the head index and the \texttt{tail+1} index respectively. One leaves \texttt{Q[Q.tail]} empty to insert new values, and delete from \texttt{Q[Q.head]}}

		\subsubsection{Operations}

				%% lstlisting %%

		\subsubsection{analysis}
			\begin{itemize}
				\item[] Memory : $O(n)$
				\item[] Running Time : $O(1)$
			\end{itemize}	

		\rem{Queue max-size defined \ita{a priori}}

		\subsection{Implementation : Linked List}

		\par{Take \texttt{L.head == Q.head} and \texttt{L.tail == Q.tail} then, for a queue to align its front with the front of the list, and similarly their back then \texttt{ENQUEUE, DEQUEUE} are implemented via \texttt{INSERT@TAIL, DELETE@HEAD} respectively}

		\subsection{Analysis}
				\begin{itemize}
						\item[] Running Time \mymarginpar{\texttt{SIZE} will need to keep track of \texttt{Q.tail}}: $O(1)$
				\end{itemize}

		\rem{Doubly-linked lists offer the same running time, but at a higher memory cost}

\section{Double Ended Queue}

		\par{Structures similar to queues which support data insertion and deletion at both the start and end}

		\rem{\ita{deque} read as \ita{"deck"}}

		\subsection{Operations}
			\begin{itemize}
						\item[] Update Methods
				\begin{enumerate}
						\item addFirst
						\item addLast
						\item delFirst
						\item delLast
				\end{enumerate}
						\item[] Accessor Methods
				\begin{enumerate}
						\item first
						\item last
						\item size
						\item isEmpty
				\end{enumerate}
			\end{itemize}
		
		\subsection{Analysis}
				\par{All operations run on $O(1)$, and depending on their implementation $O(N)$ for an array of size $N$ and $O(n)$ , where $n < N$ is the actual number of elements in a doubly linked list}


