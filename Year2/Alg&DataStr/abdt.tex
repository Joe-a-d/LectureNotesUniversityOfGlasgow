\section{Abstract Data Types}

\defn{ADT}{ is a mathematical model of a data structure that specifies the type of data stored, the operations supported on them, and the types of parameters of the operations}

\par{ADTs are abstractions of the structure of data from the data itself, they define behaviour and state, a contract whose concrete instances must follow. When implemented in code we do so via \ita{data structures}. For example, a \texttt{List} can be implemented as an \texttt{array}}

\rem{In \texttt{Java} you can think in terms of \texttt{Interfaces} and \texttt{Classes}}

		\subsection{Stack}

				\defn{Stack}{is a collection of objects that are inserted and removed following the \ita{LIFO} principle}

				\rem{Web browsers' histories use stacks, and undo sequences in most other software work in a similar manner}
  
		\subsubsection{Operations}

				\begin{itemize} 
								\item Essential Update Methods 
						\begin{enumerate}
								\item push
								\item pop
						\end{enumerate}
				\end{itemize}
				\begin{itemize}
								\item Accessor Methods
						\begin{enumerate}
								\item top : "pop" without removing
								\item size
								\item isEmpty
						\end{enumerate}
				\end{itemize}

		\subsubsection{Implementation - Array}

				\par{The array implementation of a stack is simple and efficient \ita{if} one has a good idea , in advance , of the number of elements it will contain. Then, given $n,t$ where $n$ represents the size of the stack and $t$ is a cursor var which keeps track of the index of top element, one adds elements from [0...n-1=t]}

		\rem{by convention , $ S = \emptyset \iff t = -1 $}

		\defn{Stack Overflow}{ push() into full stack}
		\defn{Stack Underflow}{ pop() empty stack}

		\subsubsection{Operations}
			\begin{algorithm}[H]
				\DontPrintSemicolon
				\SetAlgoLined\KwData{Stack $S$}
				\SetKwFunction{KwFn}{STACK-EMPTY}
				\KwResult{true, false}
				\KwFn{S}\\
				\Indp\Return S.top = -1

				\caption{isEmpty}

			\end{algorithm}

			\begin{algorithm}[H]
				\DontPrintSemicolon
				\SetAlgoLined\KwData{Stack $S$ , $x$}
				\SetKwFunction{KwFn}{PUSH}
				\KwFn{S,$x$}{\\
				\Indp S.top := S.top + 1 \;
				\Indp S[S.top] := x 
				}
				\caption{Push}

			\end{algorithm}

		\begin{algorithm}[H]
				\DontPrintSemicolon
				\SetAlgoLined\KwData{Stack $S$}
				\SetKwFunction{KwFn}{POP}
				\KwResult{"underflow", S[S.top + 1]}
				\KwFn{S}\\
				\uIf{STACK-EMPTY(S)}
					{return "underflow"}
				\uElse{
					S.top := S.top - 1 \;
					\Return S[S.top + 1]
					}

				\caption{Pop}

			\end{algorithm}

		\subsubsection{Analysis}

				\par{Each method executes a constant number of statements involving
arithmetic operations, comparisons, and assignments, or calls to size and isEmpty,
which both run in constant time, i.e they're all $O(1)$, and memory is $O(n)$}

		\subsubsection{Implementation - Dynamic Array}

				\par{One can overcome the fixed size constraint by adding a \texttt{RESIZE} operation, allowing the array to both shrink and grow.}
				\par{Simple implementations are doubling its size when full, and half it when it is one quarter full. So, when an overflow is detected the following happens:}
				\begin{enumerate}
						\item Allocate a new array $S'$ with larger capacity
						\item Set $S'[k] = S[k] , k = 0 , \dots , n-1$
						\item Set $S = S'$
				\end{enumerate}

		\subsubsection{Operations}
						\begin{algorithm}[H]
				\DontPrintSemicolon
				\SetAlgoLined\KwData{Stack $S$ , new capacity $n'$}
				\SetKwFunction{KwFn}{RESIZE}
				\KwFn{S, n'}\\
				\Indp new S’[0..n’-1] \;
				\For{i = 0 to S.top}
					{S’[i] := S[i]}
				S := S'
				\caption{Resize}

			\end{algorithm}

			\begin{algorithm}[H]
				\DontPrintSemicolon
				\SetAlgoLined\KwData{Stack $S$ , $x$}
				\SetKwFunction{KwFn}{PUSH}
				\KwFn{S,$x$}{\\
				\uIf{S.top = n -1}
					{RESIZE(S, 2*n) {\tcc*[r]{double array if full}}
					}
				S.top := S.top + 1 \;
				S[S.top] := x 
				}
				\caption{Push}

			\end{algorithm}

		\begin{algorithm}[H]
				\DontPrintSemicolon
				\SetAlgoLined\KwData{Stack $S$}
				\SetKwFunction{KwFn}{POP}
				\KwResult{"underflow", S[S.top + 1]}
				\KwFn{S}\\
				\uIf{STACK-EMPTY(S)}
					{return "underflow"}
				\uElse{
					x := S[S.top]\;
					S.top := S.top - 1 \;
					\If{S.top > 0 and S.top = n/4}
						{RESIZE(S,n/2){\tcc*[r]{half array if 1/4 full}}
					}
					\Return S[S.top + 1]
				}

				\caption{Pop}

			\end{algorithm}

		\subsubsection{Analysis}
						\begin{itemize}
								\item[] Memory : $O(kn)$
								\item[] \mymarginpar{see L.8.35 for amortised analysis}Running Time : O(n) , since when expanding by a constant proportion each insertion takes \ita{amortised} constant time
						\end{itemize}

		\subsubsection{Implementation - Linked List}
				\par{Let \texttt{l.head := S.top} , then \texttt{PUSH = INSERT@Head} , \texttt{POP = DEL@Head} and by keeping track of size with \texttt{S.size} we can perform \texttt{SIZE} at constant time}

		\subsubsection{Operations}

			\begin{algorithm}[H]
				\DontPrintSemicolon
				\SetAlgoLined\KwData{Stack $S$ , $x$}
				\SetKwFunction{KwFn}{PUSH}
				\KwFn{S,$x$}{\\
				\Indp x.next := S.top {\tcc*[r]{S.top == L.head}} \;
				\Indp S[S.top] := x 
				}
				\caption{Push}

			\end{algorithm}

		\begin{algorithm}[H]
				\DontPrintSemicolon
				\SetAlgoLined\KwData{Stack $S$}
				\SetKwFunction{KwFn}{POP}
				\KwResult{"underflow", x}
				\KwFn{S}\\
				\uIf{S.top != NILL}
					{x := S.top \;
					 S.top := S.top.next \;
					 \Return x
					}
				\uElse{
					\Return "underflow"
					}

				\caption{Pop}
		\end{algorithm}
	
		\subsubsection{Analysis}
						\begin{itemize}
								\item[] Memory : $O(n)$
								\item[] Running Time : O(1)
						\end{itemize}

