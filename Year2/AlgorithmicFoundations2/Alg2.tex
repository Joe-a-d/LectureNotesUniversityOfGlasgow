
%% CLASS MANUAL FOUND IN http://blog.poormansmath.net/latex-class-for-lecture-notes/ %%
%% CLASS AUTHOR Stefano Maggiolo %%
\documentclass[english,course]{Notes}

\title{COURSE CODE}
\subject{SUBJECT AREA}
\author{Joao Almeida-Domingues}
\email{2334590D@student.gla.ac.uk}
\speaker{LECTURER}
\date{07}{07}{1993}
\dateend{07}{07}{1994}
\place{University of Glasgow}

 %%%%% GENERAL MATHEMATICAL NOTATION SHORTCUTS %%%%%
 
\newcommand{\n}{\mathbb{N}}
\newcommand{\z}{\mathbb{Z}}
\newcommand{\q}{\mathbb{Q}}
\newcommand{\cx}{\mathbb{C}}
\newcommand{\real}{\mathbb{R}}
\newcommand{\field}{\mathbb{F}}
\newcommand{\ita}[1]{\textit{#1}}
\newcommand{\oneton}{\{1,2,3,...,n\}}
\newcommand\ef{\ita{f} }

\newcommand\inv[1]{#1^{-1}}
\newcommand\setb[1]{\{#1\}}
\newcommand\en{\ita{n }}
\renewcommand\qedsymbol{QED} %QED instead of square



%%%%%%%%%%%%%%%%PACKAGES%%%%%%%%%%%%%%%%%%%%%%%%%%%%%
%\usepackage{lipsum}  

\usepackage{amsmath,amsthm,amssymb,graphicx,mathtools,tikz,pgfplots} %maths
\usepackage{hyperref,framed,color,fancybox} %layout

\renewcommand{\abstractname}{\vspace{3\baselineskip}} %hack to remove abstract

% framed :  \begin{shaded,frame,snugshade or leftbar} \definecolor{shadecolor}{rgb}{XYZ} to change color
%fancybox: \shadowbox,ovalbox or doublebox
%\extra for Extra content layout box
%%%%%%%%%%%%%%%%%%%%%%%%%%%

%%%CLASS SHORTCUTS%%%%
%\lecture{day}{month}{year} for margin note 
%\begin{theorem} sdfsdf\end{theorem}  --> \theorem
%\begin{proposition} dfsdfs\end{proposition} --> \prop
%\begin{lemma} dsfsd \end{lemma} --> \lem
%\begin{corollary} f ffew \end{corollary}
%\begin{definition} fwewef w \end{definition} --> \defn
%\begin{example} feww e\end{example} --> \ex
%\begin{exercise} wefwe \end{exercise}
%\begin{remark} wef we \end{remark} --> \rem
%\begin{fact} wefe \end{fact}
%\begin{problem} wef ew \end{problem}
%\begin{conjecture} ewfew \end{conjecture}
%\begin{claim} few w \end{claim}
%\begin{notation} fewf \end{notation} --> \nota
%\mymarginpar for scriptsize margin

\begin{document}

\begin{abstract}
\par{These lecture notes were collated by me from a mixture of sources , the two main sources being the lecture notes provided by the lecturer and the content presented in-lecture. All other referenced material (if used) can be found in the \ita{Bibliography} and \ita{References} sections.}
\par{The primary goal of these notes is to function as a succinct but comprehensive revision aid, hence if you came by them via a search engine , please note that they're not intended to be a reflection of the quality of the materials referenced or the content lectured.}
\par{Lastly, with regards to formatting, the pdf doc was typeset in \LaTeX , using a modified version of Stefano Maggiolo's \href{http://blog.poormansmath.net/latex-class-for-lecture-notes/}{\underline{\textcolor{blue}{class}}}}
\end{abstract}

\newpage

\section{Propositional Logic}

\defn{Propositional Logic}{the logic of compound statements built from simpler statements using Boolean connectives}

\defn{Propostion}{declarative sentence which is either true or false}


\section{TODO}
\par{See slide annotations}
\par{Truth table for all connectives , use the succinct style, where for large statements the result is written under the connective}
\par{Write a remark explaining the use of the succinct form of the tables for large compound propositions (e.g. (p > q) & (~p v q) one would not make a separate column for the conjunction, but instead just write the result under the &)}
\par{Write a remark explaining that for any n propositions, rows = 2^{n}, since for every p {1,0}}


\newpage
\section{BIBLIO: ADD TO BIBTEX}

Truth Tables : Michael Rieppel ``mrieppel '' @https://github.com/mrieppel/TruthTableGenerator


\end{document}

\end{document}