
%% CLASS MANUAL FOUND IN http://blog.poormansmath.net/latex-class-for-lecture-notes/ %%
%% CLASS AUTHOR Stefano Maggiolo %%
\documentclass[english,course]{Notes}

\title{2B : Linear Algebra}
\subject{Mathematics}
\author{Joao Almeida-Domingues}
\email{2334590D@student.gla.ac.uk}
\speaker{Dr. Chris Athorne}
\date{23}{09}{2019}
\dateend{04}{12}{2019}
\place{University of Glasgow}

 %%%%% GENERAL MATHEMATICAL NOTATION SHORTCUTS %%%%%
 
\newcommand{\n}{\mathbb{N}}
\newcommand{\z}{\mathbb{Z}}
\newcommand{\q}{\mathbb{Q}}
\newcommand{\cx}{\mathbb{C}}
\newcommand{\real}{\mathbb{R}}
\newcommand{\field}{\mathbb{F}}
\newcommand{\ita}[1]{\textit{#1}}
\newcommand{\oneton}{\{1,2,3,...,n\}}
\newcommand\ef{\ita{f} }

\newcommand\inv[1]{#1^{-1}}
\newcommand\setb[1]{\{#1\}}
\newcommand\en{\ita{n }}
\renewcommand\qedsymbol{QED} %QED instead of square
\newcommand\handleft{\HandCuffLeft}
\newcommand\handright{\HandCuffRight}

%MATRICES
\usepackage[delims={[]}]{spalign}
\let\mat=\spalignmat
\let\amat=\spalignaugmat
\let\vec=\spalignvector

% row ops
\newcommand\ro[2]{\xrightarrow[#2]{#1}}

\setlength\parindent{0pt}
%%%%%%%%%%%%%%%%PACKAGES%%%%%%%%%%%%%%%%%%%%%%%%%%%%%
%\usepackage{lipsum}  

\usepackage{amsmath,amsthm,amssymb,graphicx,mathtools,tikz,pgfplots} %maths
\usepackage{hyperref,framed,color,fancybox} %layout
\usepackage[backend=biber, style=reading]{biblatex} %bibliography
\bibliography{} %add bib file name

\renewcommand{\abstractname}{\vspace{3\baselineskip}} %hack to remove abstract
\usepackage{bbding} %dingbats  for signposting

% framed :  \begin{shaded,frame,snugshade or leftbar} \definecolor{shadecolor}{rgb}{XYZ} to change color
%fancybox: \shadowbox,ovalbox or doublebox
%\extra for Extra content layout box
%%%%%%%%%%%%%%%%%%%%%%%%%%%

%%%CLASS SHORTCUTS%%%%
%\lecture{day}{month}{year} for margin note 
%\begin{theorem} sdfsdf\end{theorem}  --> \theorem
%\begin{proposition} dfsdfs\end{proposition} --> \prop
%\begin{lemma} dsfsd \end{lemma} --> \lem
%\begin{corollary} f ffew \end{corollary}
%\begin{definition} fwewef w \end{definition} --> \defn
%\begin{example} feww e\end{example} --> \ex
%\begin{exercise} wefwe \end{exercise}
%\begin{remark} wef we \end{remark} --> \rem
%\begin{fact} wefe \end{fact}
%\begin{problem} wef ew \end{problem}
%\begin{conjecture} ewfew \end{conjecture}
%\begin{claim} few w \end{claim}
%\begin{notation} fewf \end{notation} --> \nota
%\mymarginpar for scriptsize margin

\begin{document}


\begin{abstract}
\par{These lecture notes were collated by me from a mixture of sources , the two main sources being the lecture notes provided by the lecturer and the content presented in-lecture. All other referenced material (if used) can be found in the \ita{Bibliography} and \ita{References} sections.}
\par{The primary goal of these notes is to function as a succinct but comprehensive revision aid, hence if you came by them via a search engine , please note that they're not intended to be a reflection of the quality of the materials referenced or the content lectured.}
\par{Lastly, with regards to formatting, the pdf doc was typeset in \LaTeX , using a modified version of Stefano Maggiolo's \href{http://blog.poormansmath.net/latex-class-for-lecture-notes/}{\underline{\textcolor{blue}{class}}}}
\end{abstract}
\newpage


\section{Vectors}

\subsection{Basics}

\mymarginpar{Most of this has been covered already, if too succinct check 1R/1S notes}

\defn{Vector}{ displacement from one point to another in space ; geometrically represented by a directed line segment}

\rem{In general we use the origin of the cartesian coordinate system as the displacement origin}

\defn{$\real^{n}$}{ the set of all $n$-tuples of real number , for $n \in \real$ $$\real^{n} = \set{(x_{1}, \dots , x_{n}) : x_{1} , \dots, x_{n} \in R}$$}

\rem{remember that $( )$ are used for ordered objects}

\notation{$\vc{v} \quad ; \quad  \underline{v} \qquad \quad ; \quad \vc{v} = \mat{v_{1};\vdots;v_{n}} ; \quad \vc{v} = \mat{v_{1}\dots \ v_{n}}$}


\defn{Vector Addition}{$\vc{u} + \vc{v} = (u_{1} + v_{1} , \dots , u_{n} + v_{n})$}

\defn{Scalar Multiplication}{$\lambda\vc{v} = (\lambda v_{1}  , \dots ,  \lambda v_{n})$}

\begin{itemize}
\item []Algebraic Properties of Vector Addition
$$\begin{array}{l}{\mathbf{u}+\mathbf{v}=\mathbf{v}+\mathbf{u} \scriptsize\text { (commutativity of vector addition) }} \\ {(\mathbf{u}+\mathbf{v})+\mathbf{w}=\mathbf{u}+(\mathbf{v}+\mathbf{w}) \scriptsize\text { (commutativity of vector addition) }} \\ {\mathbf{u}+\mathbf{0}=\mathbf{u}} \\ {\mathbf{u}+(-\mathbf{u})=\mathbf{0}} \\ {c(\mathbf{u}+\mathbf{v})=c \mathbf{u}+c \mathbf{v} \scriptsize\text { (distributivity of vector addition) }} \\ {(c+d) \mathbf{u}=c \mathbf{u}+d \mathbf{u} \scriptsize\text { (distributivity of scalar addition) }} \\ {c(d \mathbf{u})=(c d) \mathbf{u}} \\ {\mathbf{1} \mathbf{u}=\mathbf{u}}\end{array}
$$\end{itemize}

\subsection{Linear Combination and Independence}

\defn{Linear Combination}{Sum of the members of a set, where each member is multiplied by a constant \label{genLC}}

\par{It follows from~\ref{genLC} that for a set composed of vectors , we can say that an arbitary vector $\vc{v}$ is a linear combination of $\vc{u} , \vc{w}$ \ita{iff}}

$$ \vc{v} = j\vc{u} + k\vc{w} \quad , \quad \text{ for } j, k \in \real~\label{lc}$$

\subsubsection{Systems of Linear Equations and Matrices}

\par{Note that for a vector in $\real^{n}$ we have a linear equation in $n$ variables whose solution is  a vector of size $n$. In other words, the linear equation $a_{1} x_{1}+a_{2} x_{2}+\cdots + a_{n} x_{n}=b$ has solution $a_{1} s_{1}+a_{2} s_{2}+\cdots +a_{n} s_{n}=b$}

\par{Hence, we can find the values which satisfy~\ref{lc} by finding the solution vector which satisfies all linear equations simultaneously, i.e  by constructing and solving the appropriate system of equations. This becomes clearer if we rewrite~\ref{lc} in vector form }

$$ \mat{v_{1};\vdots;v_{n}} = j\mat{u_{1};\vdots;u_{n}} + k\mat{w_{1};\vdots;w_{n}} $$

\par{Recall also that we can transform the above in an augment matrix, where each column represents a vector and its unknown coefficient, and by performing successive EROs we can simplify the system enough so as to hopefully be able to glean the solution vector from the matrix}

\defn{Pivot}{leading non-zero entry}

\defn{Row Echelon Form}{Matrix which satisfies the following:}
\begin{minipage}{\linewidth}
      \centering
      \begin{minipage}{0.45\linewidth}
      	\begin{enumerate}
		\item All pivots in lower rows are strictly to the right of those of the rows above it
		\item All non-zero rows are above zero rows
	\end{enumerate}
       \end{minipage}
      \hspace{0.05\linewidth}
\begin{minipage}{0.45\linewidth}
	$$\mat{1,2,5;0,1,3;0,0,5}$$
\end{minipage}
\end{minipage}

\defn{Reduced Row Echelon Form}{Matrix which is in REF and}
\begin{minipage}{\linewidth}
      \centering
      \begin{minipage}{0.45\linewidth}
      	\begin{enumerate}
		\item All pivots in lower rows are equal to 1
		\item Every element above the pivots is equal to 0
	\end{enumerate}
       \end{minipage}
      \hspace{0.05\linewidth}
\begin{minipage}{0.45\linewidth}
	$$\mat{1,0,0;0,1,0;0,0,1}$$
\end{minipage}
\end{minipage}


\[            \def\spalignendline{\cr}\spalignrun{\bordermatrix{\the\spaligntoks}}{,            j           k;                        ,1,            0; ,                        0            1}            \]

\newpage
\nocite{*}
\printbibliography

\end{document}