
%% CLASS MANUAL FOUND IN http://blog.poormansmath.net/latex-class-for-lecture-notes/ %%
%% CLASS AUTHOR Stefano Maggiolo %%
\documentclass[english,course]{Notes}

\title{MATHS 2P : Graphs \& Networks}
\subject{Discrete Mathematics}
\author{Joao Almeida-Domingues}
\email{2334590D@student.gla.ac.uk}
\speaker{Professor Tara Brendle}
\date{25}{09}{2019}
\dateend{04}{12}{2019}
\place{University of Glasgow}

 %%%%% GENERAL MATHEMATICAL NOTATION SHORTCUTS %%%%%
 
\newcommand{\n}{\mathbb{N}}
\newcommand{\z}{\mathbb{Z}}
\newcommand{\q}{\mathbb{Q}}
\newcommand{\cx}{\mathbb{C}}
\newcommand{\real}{\mathbb{R}}
\newcommand{\field}{\mathbb{F}}
\newcommand{\ita}[1]{\textit{#1}}
\newcommand{\oneton}{\{1,2,3,...,n\}}
\newcommand\ef{\ita{f} }
\newcommand\inv[1]{#1^{-1}}
\newcommand\setb[1]{\{#1\}}
\newcommand\en{\ita{n }}
\renewcommand\qedsymbol{QED} %QED instead of square
\newcommand\placeholder[1]{\begin{center}\includegraphics[width=#1\textwidth]{example-image-a}\end{center}}


%%%%%%%%%%%%%%%%PACKAGES%%%%%%%%%%%%%%%%%%%%%%%%%%%%%
%\usepackage{lipsum}  
\usepackage{todo}
\usepackage{amsmath,amsthm,amssymb,graphicx,mathtools,tikz,pgfplots} %maths
\usepackage{hyperref}
%\hypersetup{param1,param2,...} %overrides hyperreff package options setup in class
\usepackage{framed,color,fancybox} %layout
\usepackage{cleveref}
\usepackage{algorithm2e}
\renewcommand{\abstractname}{\vspace{2\baselineskip}} %hack to remove abstract

% framed :  \begin{shaded,frame,snugshade or leftbar} \definecolor{shadecolor}{rgb}{XYZ} to change color
%fancybox: \shadowbox,ovalbox or doublebox
%\extra for Extra content layout box
%%%%%%%%%%%%%%%%%%%%%%%%%%%

%%%CLASS SHORTCUTS%%%%
%\lecture{day}{month}{year} for margin note 
%\begin{theorem} sdfsdf\end{theorem}  --> \theorem
%\begin{proposition} dfsdfs\end{proposition} --> \prop
%\begin{lemma} dsfsd \end{lemma} --> \lem
%\begin{corollary} f ffew \end{corollary}
%\begin{definition} fwewef w \end{definition} --> \defn
%\begin{example} feww e\end{example} --> \ex
%\begin{exercise} wefwe \end{exercise}
%\begin{remark} wef we \end{remark} --> \rem
%\begin{fact} wefe \end{fact}
%\begin{problem} wef ew \end{problem}
%\begin{conjecture} ewfew \end{conjecture}
%\begin{claim} few w \end{claim}
%\begin{notation} fewf \end{notation} --> \nota
%\mymarginpar for scriptsize margin
% bold vectors --> \vc
% Placeholder figure -> \palceholder
\begin{document}

\begin{abstract}
\par{These lecture notes were collated by me from a mixture of sources , the two main sources being the lecture notes provided by the lecturer and the content presented in-lecture. All other referenced material (if used) can be found in the \ita{Bibliography} and \ita{References} sections.}
\par{The primary goal of these notes is to function as a succinct but comprehensive revision aid, hence if you came by them via a search engine , please note that they're not intended to be a reflection of the quality of the materials referenced or the content lectured.}
\par{Lastly, with regards to formatting, the pdf doc was typeset in \LaTeX , using a modified version of Stefano Maggiolo's \href{http://blog.poormansmath.net/latex-class-for-lecture-notes/}{\underline{\textcolor{blue}{class}}}}
\end{abstract}

\newpage

\section{Fundamentals}
\lecture{25}{09}{2019}
\subsection{Graphs}

\par{A graph $G$ is a pair $(V,E)$, where $V$ is any finite set, and $E$ is a set whose elements are pairs of elements of $V$. We call the elements of $V$ the \ita{vertices}*\mymarginpar{* often also referred as nodes} of $G$ and those of $E$ its \ita{edges}. e.g. $G=\set{\set{a,b,c} , \set{ab,ac}}$}

\defn{Adjacent Vertices}{are vertices connected directly through an edge. Formally, if  $e = \set{u,v} \in E$ , then $u , v$ are adjacent}

\defn{Incident Edges}{are edges which share a vertex. We say that they are \ita{``incident to v''}}

\lecture{27}{09}{19}
\subsubsection{Representing Graphs}

\begin{enumerate}
\item[] Pictorially \mymarginpar{Note that the representation need not be unique}
	\placeholder{0.4}
	
\item[] Adjacency Matrix
 
 \mymarginpar{Note that this definition only holds for simple graphs, i.e without loops. But , it is easily generalised if the binary requirement is dropped} 
 \defn{Adjacency Matrix}{ is the $ n \times n$ binary matrix , where $n = |V|$ and $a_{ij} =1 \iff e = \set{u,v} \in E$ ; i.e iff $u , v$ are adjacent }
 
 $$ A= \left(\begin{matrix}0&1&1&1\\1&0&1&0\\1&1&0&1\\1&0&1&0\end{matrix}\right) $$
 
 \rem{In this course, we'll only deal with \ita{simple}, \ita{undirected} graphs. Note that AMs of this type have the nice property of being symmetric (see 2B notes for properties)}
\end{enumerate}

\subsubsection{Subgraphs}

\defn{Subgraphs}{ are graphs obtained by deleting edges and/or edges of another graph}

\defn{Induced Subgraph}{ is a graph formed by deleting only nodes and their incident edges. Formally: Let $W \subset V$, then the induced subgraph of G is given by $G[W] = \set{W , \set{\set{xy} | xy \in G }}$. We say that \ita{``G is induced by W''} }


\example{ 

$$ G(V) = \set{ \set{a,b,c} , \set{ab, ac}} \text{ and } U = \set{c}  \text{ then } G[U] = \set{\set{a,b,\cancel{c}} ,\set{ab,\cancel{ac}}}$$}

\defn{Spanning Subgraph}{ similar to the induced, but edges are deleted instead}

\lecture{2}{10}{19}
\subsection{Graph Properties}

\defn{Walk}{from $u$ to $v$ is a sequence of vertices $w1, \dots ,wp$ (for some natural number $p \geq 2$), with $w_{1} = u$ and $w_{p} = v$, such that $w_{i}w_{i+1}$ is an edge for every $1 \leq i \leq p - 1$\label{1:walk}}

\par{Informally, a walk is just a sequence of vertices, where each subsequent vertex added to the sequence forms an edge with the preceding one}

\defn{Trail}{a walk with distinct edges}

\defn{Path}{a walk with distinct vertices}

\rem{In general, every walk  between two vertices contains a path \todo*{Remove commented out cits}}%\cite{Diestel}} 

\example{For a graph $P(\set{a,b,c,d,e,f} , \set{ab,ac,ad,bc,bd,cd,de,ef,}$) , ~\todo*{Add pictorial representation}

\begin{itemize}
\item[]\textbf{Walk: } $W = \set{abcdeacd}$ 
\item[] \textbf{Trail :} $T = \set{abcdea} = W \setminus \set{cd}_{2} $ 
\item[] \textbf{Path: } $P = \set{abcde} = W \setminus \set{acd}_{2}$ 
\mymarginpar{where $\set{x}_{2}$ is improper notation for the repeated instances of x in a set}
\end{itemize}
}
\prop{Number of Paths }{For a graph with $n$ vertices, there are $(n-1)^{n}$ paths}~\todo*{proof paths}
\defn{Connected}{A graph $G = (V, E) $ is connected if, for every two distinct vertices $u , v \in V$, there is a path in $G$ from $u$ to $v$}

\rem{A single vertex graph is connected. Since it has not distinct vertices, we say that the definition holds \ita{vacuously}}

\defn{Connected Component}{$H$ is a connected 
component of $G$ if $H$ is a connected induced subgraph of $G$ and, for
any subgraph $H'$ of $G$ such that $V(H) \subset V(H'), H'$ is not connected}

\rem{The vertex sets of distinct connected components are necessarily disjoint}

\placeholder{0.4}

\defn{Vertex Degree}{$d(v) = |E(v)|$ , i.e. it is the size of the set of all edges connected to $v$}

\defn{Minimum Degree}{$\min _{v \in V(G)} d(v)$ , i.e. a graph's minimum degree is equal to the lowest degree of its vertices} \mymarginpar{The converse is true of the maximum}

\subsection{Isomorphisms}

\defn{Isomorphism}{from $G_{1} = (V_{1}, E_{1})$ to $G_{2} = (V_{2}, E_{2})$ is a
bijection $f : V_{1} \rightarrow V_{2}$ such that, for every $u, v \in V_{1}, f(u) f(v) \in E_{2}
\equiv uv \in E_{1}$}

\rem{Specifically, we can consider $f$ to be a process whereby one \ita{relabels} the vertices}

\placeholder{0.5}

\todo{methods to determine isomorphism}

\section{Special Graphs}

\defn{Complete Graphs}{Every pair of distinct vertices forms an edge}

\notation{$K_{n}$ , for a graph with $n$ vertices}

\prop{Number of Edges}{ \ $K_{n}$ has $\frac{1}{2}n(n-1)$ edges}~\todo{proof \#edges}

\defn{Paths}{a path on n vertices is a graph that is isomorphic to the graph $(V, E)$ where $V = \set{v_{1}, \dots, v_{n}}$ and $E = \set{v_{i}v_{i+1}}: 1 \leq i \leq n - 1$}
\notation{$P_{n}$}

\rem{$P_{n} $ has $n-1$ edges}

\defn{Cycle}{a cycle on $n$ vertices is a graph that is isomorphic to the graph $(V, E)$ where $V=\set{v_{1}, \ldots, v_{n}} \text{ and } E=\set{v_{i} v_{i+1}}: 1 \leq i \leq n-1 \cup \set{v_{n} v_{1}}$. i.e, it's a path with the end vertices connected}

\notation{$C_{n}$}

\section{Trees}

\defn{Forest}{an \ita{acyclic} graph, i.e. without cycles}

\defn{Tree}{connected acyclic graph}

\defn{Leaf}{vertex of degree 1}

\placeholder{0.5} \todo{Take fig.1 from lectures and connect the adjacent endpoints of all trees do contrast with forest}

\subsection{Basic Properties}

\prop{Leafs }{Every tree as at least one leaf}

\prop{Number of Edges }{$T_{n} \implies |E(T)| = n-1$}

\prop{Connected Graph }{Every connected graph with $n$ vertices and $n-1$ edges is a Tree}

\prop{Forests }{For a forest $F_{n}$ with $c$ connected components , $|E(T)| = n-c$}

\rem{Hence, note that a tree is just a special case of a forest, where $c=1$}

\subsection{Spanning Trees}

\defn{Spanning Tree}{Spanning subgraph which is not a tree}~\todo{Add example of sp.sub  is vs is not}

\placeholder{0.5}

\prop{Necessity }{Every connected graph contains a spanning tree}
as
\theorem{Cayley's Formula : A complete graph with $n$ vertices has $n^{n-2}$ (labelled) spanning trees}~\label{2:caley}

\par{It follows from \ref{2:caley} that , if we're interested in finding a \ita{minimum spanning tree} (a weighted sp.tree of minimum weight), an exhaustive search through all possible trees becomes a gruelling task very quickly. There are however two \ita{greedy} algorithms which help}

\subsection{Kruskal's Algorithm}

\par{For every edge not in the tree, add the one which has minimum weight and does not form a cycle. Stop when connected}

\begin{algorithm}[H]
\SetAlgoLined\KwData{this text}
\KwResult{how to write algorithm with \LaTeX2e }initialization\;
	\While{not at end of this document}{read current\;
		\eIf{understand}{go to next section\;
			current section becomes this one\;
		}{go back to the beginning of current section\;
		}
	}
\caption{How to write algorithms}
\end{algorithm}

\todo*{Add Kruskal's typesetting. Change label to header}
\todo*{Fix broken layout, where theorem environment after caley's is spilling over the rest of the text}



\theorem{Kruskal's }{will always output a M.S.T}

\proofs{\mymarginpar{Reproduction not examinable, merely analysis}}

\subsection{Prim's}

\par{Similar to Kruskal's but instead of looking for $min(E)$, we look for the smallest which adjacent to a node in the last iteration}

\section{Proof Techniques}



\newpage

\todo*{BibTex : Diestel,Reinhard ; Graph Theory}
\todo*{Adjust class to remove italics from prop}
\todos

\end{document}

