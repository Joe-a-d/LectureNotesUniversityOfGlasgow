
%% CLASS MANUAL FOUND IN http://blog.poormansmath.net/latex-class-for-lecture-notes/ %%
%% CLASS AUTHOR Stefano Maggiolo %%
\documentclass[english,course]{Notes}

\title{MATHS 2P : Graphs \& Networks}
\subject{Discrete Mathematics}
\author{Joao Almeida-Domingues}
\email{2334590D@student.gla.ac.uk}
\speaker{Professor Tara Brendle}
\date{25}{09}{2019}
\dateend{04}{12}{2019}
\place{University of Glasgow}

 %%%%% GENERAL MATHEMATICAL NOTATION SHORTCUTS %%%%%
 
\newcommand{\n}{\mathbb{N}}
\newcommand{\z}{\mathbb{Z}}
\newcommand{\q}{\mathbb{Q}}
\newcommand{\cx}{\mathbb{C}}
\newcommand{\real}{\mathbb{R}}
\newcommand{\field}{\mathbb{F}}
\newcommand{\ita}[1]{\textit{#1}}
\newcommand{\oneton}{\{1,2,3,...,n\}}
\newcommand\ef{\ita{f} }
\newcommand\inv[1]{#1^{-1}}
\newcommand\setb[1]{\{#1\}}
\newcommand\en{\ita{n }}
\renewcommand\qedsymbol{QED} %QED instead of square
\newcommand\placeholder[1]{\begin{center}\includegraphics[width=#1\textwidth]{example-image-a}\end{center}}


%%%%%%%%%%%%%%%%PACKAGES%%%%%%%%%%%%%%%%%%%%%%%%%%%%%
%\usepackage{lipsum}  

\usepackage{amsmath,amsthm,amssymb,graphicx,mathtools,tikz,pgfplots} %maths
\usepackage{hyperref}
%\hypersetup{param1,param2,...} %overrides hyperreff package options setup in class
\usepackage{framed,color,fancybox} %layout
\usepackage{cleveref}

\renewcommand{\abstractname}{\vspace{2\baselineskip}} %hack to remove abstract

% framed :  \begin{shaded,frame,snugshade or leftbar} \definecolor{shadecolor}{rgb}{XYZ} to change color
%fancybox: \shadowbox,ovalbox or doublebox
%\extra for Extra content layout box
%%%%%%%%%%%%%%%%%%%%%%%%%%%

%%%CLASS SHORTCUTS%%%%
%\lecture{day}{month}{year} for margin note 
%\begin{theorem} sdfsdf\end{theorem}  --> \theorem
%\begin{proposition} dfsdfs\end{proposition} --> \prop
%\begin{lemma} dsfsd \end{lemma} --> \lem
%\begin{corollary} f ffew \end{corollary}
%\begin{definition} fwewef w \end{definition} --> \defn
%\begin{example} feww e\end{example} --> \ex
%\begin{exercise} wefwe \end{exercise}
%\begin{remark} wef we \end{remark} --> \rem
%\begin{fact} wefe \end{fact}
%\begin{problem} wef ew \end{problem}
%\begin{conjecture} ewfew \end{conjecture}
%\begin{claim} few w \end{claim}
%\begin{notation} fewf \end{notation} --> \nota
%\mymarginpar for scriptsize margin
% bold vectors --> \vc
% Placeholder figure -> \palceholder
\begin{document}

\begin{abstract}
\par{These lecture notes were collated by me from a mixture of sources , the two main sources being the lecture notes provided by the lecturer and the content presented in-lecture. All other referenced material (if used) can be found in the \ita{Bibliography} and \ita{References} sections.}
\par{The primary goal of these notes is to function as a succinct but comprehensive revision aid, hence if you came by them via a search engine , please note that they're not intended to be a reflection of the quality of the materials referenced or the content lectured.}
\par{Lastly, with regards to formatting, the pdf doc was typeset in \LaTeX , using a modified version of Stefano Maggiolo's \href{http://blog.poormansmath.net/latex-class-for-lecture-notes/}{\underline{\textcolor{blue}{class}}}}
\end{abstract}

\newpage

\section{Fundamentals}
\lecture{25}{09}{2019}
\subsection{Graphs}

\par{A graph $G$ is a pair $(V,E)$, where $V$ is any finite set, and $E$ is a set whose elements are pairs of elements of $V$. We call the elements of $V$ the \ita{vertices}*\mymarginpar{* often also referred as nodes} of $G$ and those of $E$ its \ita{edges}. e.g. $G=\set{\set{a,b,c} , \set{ab,ac}}$}

\defn{Adjacent Vertices}{are vertices connected directly through an edge. Formally, if  $e = \set{u,v} \in E$ , then $u , v$ are adjacent}

\defn{Incident Edges}{are edges which share a vertex. We say that they are \ita{``incident to v''}}

\lecture{27}{09}{19}
\subsubsection{Representing Graphs}

\begin{enumerate}
\item[] Pictorially \mymarginpar{Note that the representation need not be unique}
	\placeholder{0.4}
	
\item[] Adjacency Matrix
 
 \mymarginpar{Note that this definition only holds for simple graphs, i.e without loops. But , it is easily generalised if the binary requirement is dropped} 
 \defn{Adjacency Matrix}{ is the $ n \times n$ binary matrix , where $n = |V|$ and $a_{ij} =1 \iff e = \set{u,v} \in E$ ; i.e iff $u , v$ are adjacent }
 
 $$ A= \left(\begin{matrix}0&1&1&1\\1&0&1&0\\1&1&0&1\\1&0&1&0\end{matrix}\right) $$
 
 \rem{In this course, we'll only deal with \ita{simple}, \ita{undirected} graphs. Note that AMs of this type have the nice property of being symmetric (see 2B notes for properties)}
\end{enumerate}

\subsubsection{Subgraphs}

\defn{Subgraphs}{ are graphs obtained by deleting edges and/or edges of another graph}

\defn{Induced Subgraph}{ is a graph formed by deleting only nodes and its incident edges. Formally: Let $W \subset V$, then the induced subgraph of G is given by $G[W] = \set{W , \set{\set{xy} | xy \in G }}$. We say that \ita{``G is induced by W''} }


\example{ 

$$ G(V) = \set{ \set{a,b,c} , \set{ab, ac}} \text{ and } U = \set{c}  \text{ then } G[U] = \set{\set{a,b,\cancel{c}} ,\set{ab,\cancel{ac}}}$$}

\defn{Spanning Subgraph}{ similar to the induced, but edges are deleted instead}

\lecture{2}{10}{19}
\subsection{Graph Properties}



\end{document}

\end{document}